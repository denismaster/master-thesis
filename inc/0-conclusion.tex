\anonsection{Заключение}

В рамках данной выпускной квалификационной работы была поставлена задача разработки системы поддержки принятия решений для анализе процесса деградации комплексных технических систем, а также исследования применения сплайнов в задаче визуализации и аппроксимации данного процесса. 
Данная проблема становится все более актуальной с учетом текущего прогресса в сфере развития информационных систем и технологий в мире. 
К сожалению, большинство разработок в сфере анализа ЖЦ систем практически не упоминают процесс деградации. 
Проведенное исследование литературных источников показало, что подход с использованием сплайнов показывает хорошие результаты в других областях, однако ранее не был использован при анализе ЖЦ систем. 

В ходе выполнения поставленных задач, была проанализирована предметная область процесса деградации систем, а также структура процесса работы СППР. 
Выполнен системотехнический анализ информационной системы поддержки принятия решений.
На основании результатов анализа, а также изучения современных способов решения проблем, был сформулирован список требований к проектируемой системе, предназначение которой – визуализация и аппроксимация процесса деградации ТС с использованием сплайнов с целью повысить эффективность анализа данного процесса. 
Были определены требования к данной системе, а также определена наиболее подходящая модель качества.

Для практической реализации СППР было принято решение использовать web-технологии. 
Данная архитектура является наиболее подходящей для данной системы, что было доказано при проведении вариантного анализа. 
В качестве математической модели предложен аппарат сплайнов, а также системы уравнений для поиска наиболее оптимальных моментов реинжиниринга, поиска значения ИПЭ в заданный момент времени, а также для определения наиболее оптимальной модели с учетом заданных ограничений.

Для программной реализации сервисов были использованы современные технологи и платформы, в частности, Node.js, Nest.js, React. 
Была протестирована работоспособность приложения в различных сценариях использования.
Применение различных комбинаций технологий и подходов позволяет достичь максимального качества и соответствия требуемой функциональности.

В ходе исследования деградационных процессов были изучены такие СТО, как турбина гидроэлектростанции,
а также информационная система расчетов монетизации льгот.

Также данный метод может быть использован для определения оптимальных моментов начала модернизации системы.
В приведенном примере был произведен анализ информационной системы расчета монетизации льгот.
Определено, что документальный срок выхода системы на модернзиацию составляет 30 месяцев.
Использование системы позволило рассчитать оптимальный момент, равный 28,5 месяцам.
Также было изучено влияние различных параметров при анализе деградационных процессов данной ИС.

Был произведен анализ процесса устаревания турбин гидроэлектростанции с использованием разработанной СППР.
При исследовании различных технологий разработки турбин было определено, что наиболее оптимальной технологией является технология D,
благодаря которой турбина может оставаться работоспособной на протяжении свыше 10 лет.

Установлено, что применение сплайнов позволяет влиять на модель деградационного процесса сложного технического объекта.
Благодаря заданию опорных точек, можно изменять результирующий сплайн.
В результате можно проектировать системы, деградационные процессы которых будут развиваться по заранее заданным шаблонам.

Таким образом, достигнуты все поставленные цели и задачи данной работы. 
Был применен метод сплайнов для визуализации и аппроксимации процесса деградации ИС. 
Была разработана СППР с поддержкой многопользовательской работы. 
Разработанная система находится в открытом доступе и под открытой лицензией.


Система будет также развиваться в будущем. 
Есть возможности по расширению системы путем улучшения математических примитивов, используемых в ней.  
Применение сплайнов доказало свою состоятельность в качестве эффективного и производительного решения для поставленных задач.

% В ходе выполнения выпускной квалификационной работы магистра были исследованы процессы обеспечения безопасности облачных сред.

% В ходе исследования были проанализированы существующие проблемы и стандарты безопасности облачных вычислений, предложены способы решения данных проблем.
% Рассмотрена специфика предоставления облачных услуг зарубежных и отечественных поставщиков.
% Проанализированы наиболее опасные уязвимости за 2016~г.

% В ходе системного анализа было описано системотехническое представление системы, описаны входные и выходные данные, составлен список функций системы безопасности, произведена декомпозиция системы и описана связь между ее элементами.

% В ходе вариантного анализа был произведен сравнительный анализ гипервизоров между тремя альтернативами, в ходе которого был выбран наиболее оптимальный вариант.

% В ходе экспериментальных исследования была эксплуатирована уязвимость CVE-2016-5195, благодаря которой локальный пользователь сервера получил доступ к правам суперпользователя.
% Скрипт не является законченным продуктом и распространяется под свободной лицензией.

% Для мониторинга уязвимостей в программном обеспечении облачной среды был разработан скрипт на языке программирования Python.
% Скрипт осуществляет поиск по открытой базе уязвимостей согласно установленным параметрам.
% Программа находится в открытом доступе и распространяется под открытой лицензией.

% Практическая значимость исследования состоит в возможности применения написанной программы в облачной среде провайдеров для анализа уязвимостей и незамедлительного реагирования на них.
% Также разработана защищенная облачная инфраструктура, описаны стратегии расширения инфраструктуры.

\clearpage
