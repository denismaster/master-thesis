\section{Исследование процесса деградации сложных технических объектов с использованием системы поддержки принятия решений}

При эксплуатации сложных технических объектов возникают деградационные процессы,
в результате которых общая эффективность СТО снижается.
Данные процессы могут возникнуть в различных частях СТО, что приводит к изменению показателей критериев системы во времении.
Таким образом, при анализе процесса деградации СТО следует учитывать, каким образом параметры системы изменяются с течением времени.

Рассмотрим применение разработанной системы для анализа сложных технических объектов.

\subsection{Анализ процесса деградации информационной системы}
Рассмотрим применение системы поддержки принятия решений для анализа процесса устаревания информационной системы расчета монетизации льгот по услугам ЖКХ.
В ходе разработки данной ИС, достаточно сложно было спрогнозировать особенности ее дальнейшего развития в виду 
неравномерности финансовых вложений и затрат сил на разработку на фазе поддержки.

Для анализа данного сложного технического объекта и его процесса деградации сформируем множество скалярных критериев $K=\left\{k_1,k_2\right\}$:
\begin{itemize}
    \item $k_1$ --- доля средств, затрачиваемых на поддержку системы относительно начального бюджета;
    \item $k_2$ --- вероятность безошибочного расчета по нормативным тарифам;
    \item $k_3$ --- вероятность безошибочного ввода информации по льготам пользователя.
\end{itemize}

Рассмотрим подробнее данные критерии.
После разработки системы каждый квартал выделяется определенная сумма, предназначенная для поддержки системы.
Зависимость средств, выделяемых на поддержку, от ежеквартальных затрат на разработку, представлена на рисунке \ref{k_support}:

\addimg{k_support}{1}{Зависимость ежеквартальных средств на поддержку от средств на разработку}{k_support}

Вероятность безошибочого расчета по нормативным тарифам изменяется по экспоненциальному закону $E=E_0 \cdot e^{-0.24(t-t_0)}$.
Это связано с изменением нормативных актов, определяющих порядок расчета, а также изменением реализации модулей расчета для данных законов.

Вероятность безошибочого ввода информации по льготам пользователя изменяется по экспоненциальному закону $E=E_0 \cdot e^{-0.1(t-t_0)}$.
Это связано с увеличением числа пользователей, которые используют электронный метод ввода, а также общим процентом ошибок ввода.

Для расчета интегрального показателя эффективности используется аддитивная свертка:
\begin{center}
$K=0.2k_1+0.4k_2+0.4k_3, K \in [0,1]$
\end{center}

Минимальный уровень ИПЭ равен 0.25.

Согласно официальной документации, гарантийное время использования системы --- 10 кварталов (30 месяцев).
Рассчитаем, совпадает ли оптимальный момент реинжиниринга с требуемым, а также определим параметры, влияющие на определение ИПЭ.

График изменения ИПЭ данной системы в СППР как представлен на рисунке \ref{exp_22}
На данном рисунке красной точкой обозначается оптимальный момент модернизации, а желтой - начало деградационного процесса.
Критерии $k_2$ и $k_3$ заданы с использованием экспоненциального закона.
Критерий $k_1$ задан набором данных, поэтому используется сплайновая аппроксимация с применением кубических сплайнов.
Итоговое значение интегрального показателя эффективности рассчитывается с применением мультипликативной свертки.

\addimg{exp_22}{1}{График ИПЭ системы --- аддитивная свертка}{exp_22}

На данном рисунке видно, что оптимальным моментом реинжиниринга является значение $9.14$ кварталов, 
что равно $36.56$ месяцев. 
Согласно этому, можно сделать вывод, что оптимальные моменты не совпадают, а модернизация системы требуется раньше, чем указано в документации.
Изменим минимальный допустимый уровень эффективности на 0.45. Результат показан на рисунке \ref{exp_23}.

\addimg{exp_23}{1}{График ИПЭ системы --- аддитивная свертка}{exp_23}

Изменим тип свертки на мультипликативную. 
Результат показан на рисунке \ref{exp_33}.

\addimg{exp_33}{1}{График ИПЭ системы --- мультипликативная свертка}{exp_33}

Изменим весовые характеристики критериев на следующие, чтобы подчеркнуть важность показателей $k_2$ и $k_3$:

\begin{center}
    $w(k_1)=0.1, w(k_2)=0.45,w(k_3)=0.45$
\end{center}

Результат показан на рисунках \ref{exp_44} для минимального уровня ИПЭ 0.25, и на рисунке \ref{exp_45} для минимального уровня ИПЭ равного 0.45.

\addimg{exp_44}{1}{График ИПЭ системы с измененными весами}{exp_44}

\addimg{exp_45}{1}{График ИПЭ системы c измененными весами}{exp_45}

Исследуем динамику изменения эффективности системы с измененными данными о бюджете поддержки системы.
Пусть через некоторое время бюджет, выделенный на поддержку, был снова повышен.
Измененный график показан на рисунке \ref{k_support2}.
На данном рисунке четко прослеживается момент модернизации, когда доля выделенных на поддержку ИС средств после длительного падения вновь возрастает.

\addimg{k_support2}{1}{Измененная зависимость ежеквартальных средств на поддержку от средств на разработку}{k_support2}

Рассчитанные данные в СППР показаны на рисунках \ref{exp_55} для минимального уровня ИПЭ 0.25, и на рисунке \ref{exp_56} для минимального уровня ИПЭ равного 0.45.

\addimg{exp_55}{1}{График ИПЭ системы --- другие данные изменения доли средств}{exp_55}

\addimg{exp_56}{1}{График ИПЭ системы -- другие данные изменения доли средств}{exp_56}

Сведем полученные результаты в таблицу \ref{table:is_results}.

\begin{table}[H]
    \centering
    \caption{Анализ результатов изменения моделей ИС}\label{table:is_results}
    \begin{tabular}{|c|c|c|}
    \hline Модель & $E_m=0.25$ & $E_m=0.45$ \\
    \hline Исходная & 9.14 & 5.486 \\
    \hline С применением измененных весов & 9.329 & 5.345 \\
    \hline С применением других данных о бюджете & 9.799 & 5.587 \\
    \hline
    \end{tabular}
\end{table}

Анализируя полученные результаты, можно сделать вывод, что изменение весовых характеристик,
метода свертки, минимального допустимого уровня эффективности, а также набора данных для аппроксимации
влияет на результирующую кривую процесса деградации.

Изменение минимального допустимого уровня эффективности приводит к наступлению момента модернизации в более ранее время.
Изменение метода свертки и весовых характеристик может как приблизить оптимальный момент, так и отдалить.
Использование измененных данных с учетом добавления бюджета на разработку ИС также позволяет отдалить оптимальный момент модернизации.
В среднем оптимальный момент модернизации практически совпадает с проектным.

\subsection{Анализ процесса деградации лопаток турбин гидроэлектростанции}
В качестве примера исследован процесс деградации лопаток турбин гидроэлектростанции \cite{Degradation}.
Лопатки турбины гидроэлектростанции изготавливаются из высококачественных и износостойких сплавов.
Однако суровые условия эксплуатации, в частности, жесткость речной воды, влияют на их работу.
В лопатках турбин могут появляться трещины, которые расширяются в процессе использования.
Турбины с трещинами более 30 мм считаются непригодными к эксплуатации и подлежат замене \cite{Degradation}.

Для процесса анализа отобраны пять видов лопаток турбин, изготовленные по разной технологии и с использованием различных сплавов \cite{Degradation}.
Было проведено статистическое исследование по измерению трещины в лопатках турбин в зависимости от времени.
Усредненные результаты \cite{Degradation}. представлены в таблице \ref{table:turbine}, а также на рисунке \ref{turbine}.

\begin{table}[H]
    \centering
    \caption{Усредненные длины трещины в лопатках разных типов в зависимости от времени}\label{table:turbine}
    \begin{tabular}{|c|c|c|c|c|c|}
    \hline Время, лет & А & B & C & D & E \\
    \hline 1 & 15 & 10 & 17 & 12 & 10 \\
    \hline 3 & 20 & 15 & 25 &  16 & 15 \\
    \hline 5 & 22 & 20 & 26 & 17 & 20 \\
    \hline 7 & 26 & 25 & 27 & 20 & 26 \\
    \hline 9 & 29 & 30 & 33 & 26 & 33 \\
    \hline
    \end{tabular}
\end{table}

\addimg{turbine}{1}{Динамика расширения трещины в лопатке турбины}{turbine}

Динамика расширения трещины представлена на рисунке \ref{turbine2}.

\addimg{turbine2}{1}{Динамика расширения трещины в лопатке турбины}{turbine2}

С другой стороны, на эффективность работы также влияет просвет труб, по которым подается вода.
В данных трубах могут селиться моллюски, а также образовываться налет из минеральных солей.

Проходимость труб измеряется в процентах, а динамика изменения данного просвета в зависимости от времени представлена на рисунке \ref{turbine4}.

\addimg{turbine4}{1}{Динамика изменения проходимости труб}{turbine4}

Для анализа данного сложного технического объекта и его процесса деградации сформируем множество скалярных критериев $K=\left\{k_1,k_2\right\}$:
\begin{itemize}
    \item $k_1$ --- длина трещины;
    \item $k_2$ --- проходимость труб для подачи воды.
\end{itemize}

Длина трещины в лопатке турбины измеряется в миллиметрах.
Проходимость труб измеряется в процентах, где $100\%$ - полная проходимость, $0\%$ - полная непроходимость.

Так как критерии имеют разную структуру, выполним их нормализацию. 
Критерий $k_1$ можно нормировать путем преобразования длины в длину роста трещины до выхода лопатки из строя с последующей нормировкой.
Таким образом, можно рассчитать нормализованное значение показателя на основе формулы \ref{eq:turbine_norm}:

\begin{equation} \label{eq:turbine_norm}
    K_n(t)=(L_{max} - L)/L_{max}
\end{equation}
\vspace{0.5em}

\noindent
где $K_n(t)$ --- нормированное значение критерия,
$L$ --- текущее значение длины трещины, 
$L_{max}$ --- максимально допустимое значение длины трещины.

Нормализованный процесс устаревания лопатки турбины в виду увеличения длины трещины показана на рисунке \ref{turbine3}.

\addimg{turbine3}{1}{Нормализованный процесс устаревания лопатки турбины}{turbine3}

Критерий $k_2$ также можно привести к интервалу $[0,1]$.

Целью исследования является поиск технологии лопатки турбины, при которых оптимальный момент модернизации системы представлен позже остальных.
Выполним данное исследование с помощью разработанной СППР.
Пример задания критериев в интерфейсе системы показан на рисунке \ref{exp_1}

\addimg{exp_1}{1}{Пример отображения критериев}{exp_1}

График изменения эффективности СТО с использованием лопатки типа А показан на рисунке \ref{turbine_a}.
График изменения эффективности СТО с использованием лопатки типа B показан на рисунке \ref{turbine_b}.
График изменения эффективности СТО с использованием лопатки типа C показан на рисунке \ref{turbine_c}.
График изменения эффективности СТО с использованием лопатки типа D показан на рисунке \ref{turbine_d}.
График изменения эффективности СТО с использованием лопатки типа E показан на рисунке \ref{turbine_e}.

\addimg{turbine_a}{1}{Графи изменения эффективности при использовании лопатки типа А}{turbine_a}
\addimg{turbine_b}{1}{Графи изменения эффективности при использовании лопатки типа B}{turbine_b}
\addimg{turbine_c}{1}{Графи изменения эффективности при использовании лопатки типа C}{turbine_c}
\addimg{turbine_d}{1}{Графи изменения эффективности при использовании лопатки типа D}{turbine_d}
\addimg{turbine_e}{1}{Графи изменения эффективности при использовании лопатки типа E}{turbine_e}

Красные точки на данных графиках обозначают оптимальный момент модернизации.

Выполнив анализ данных графиков, сведем полученные результаты в таблицу \ref{table:turbine_results} и сравним с результатами, приведенными в источнике данных \cite{Degradation}.

\begin{table}[H]
    \centering
    \caption{Усредненные длины трещины в лопатках разных типов в зависимости от времени}\label{table:turbine_results}
    \begin{tabular}{|c|c|}
    \hline Тип лопатки & Оптимальный момент модернизации, лет \\
    \hline A & 9.754 \\
    \hline B & 9 \\
    \hline C & 9 \\
    \hline D & 10.278 \\
    \hline E & 8.281 \\
    \hline
    \end{tabular}
\end{table}

Таким образом, наиболее долгоживущей оказалась лопатка типа D. 
Согласно \cite{Degradation}, наиболее надежной технологией лопаток турбин также является технология D.
Таким образом, расчеты совпали.

Можно сделать вывод, что оптимальный срок службы турбины электростанции составляет около 10 лет, после чего потребуется капитальный ремонт.

\anonsubsection{Выводы раздела 4}
В ходе исследования деградационных процессов были изучены такие СТО, как турбина гидроэлектростанции,
а также информационная система расчетов монетизации льгот.
Были определены критерии и метод расчета ИПЭ. 
Также приведен пример поиска оптимального момента модернизации системы.
Определено влияние различнх параметров на расчет оптимального момента модернизации.
Также определена наиболее эффективная технология проектирования лопатки турбин электростанции.

% Рассмотрим ситуацию, представленную на рис. \ref{fig:exp_vs_curve}, где показано изменение эффективности 
% муниципальной системы, в примере системы для расчета начислений по услугам ЖКХ. 
% По оси $X$ указаны номера недель работы системы, а по оси $Y$ - эффективность системы.

% Данная система является интенсивно устаревающий.
% Такой вид процесса обусловлен разными факторами: внедряются новые тарифы и методики расчета; существует тенденция к росту граждан, обращающихся в центы расчета; аппаратное и программное обеспечение вырабатывает определенный ресурс. 
% При этом производственныеи человеческие ресурсы на фазе поддержки информационной системы ЖКХ достаточно ограничены, что влияет на снижение ее эффективности. С другой стороны, на фазе поддержки могут быть привлечены дополнительные ресурсы, в том числе специалисты в области информационных технологий, что в целом повлияет на изменение вида функции устаревания.
% Пример такого процесса может быть представлен не одим, а несколькими изгибами кривой устаревания ИС ЖКХ.
% Наличие точек перегиба является следствием попытки повлиять на жизненный цикл ИС и увеличить эффективность системы в период ее эксплуатации.
% Это возможно с привлечением дополнительных финансовых и человеческих ресурсов.

% Использование обычных функций, в частности, экспоненциальной функции $E(t)=e^{-k(t-t_m)}$  покажет лишь характер процесса.
% Коэффициент $k$, определяющий наклон кривой, определяет ее наклон с самого начала процесса. 
% Использование экспоненты показано на рис. \ref{fig:exp_vs_curve}.

% Аппроксимация подобной функции при помощи обычных нелинейных функций приведет к использованию кусочно-заданных функций, 
% что усложняет решение системы уравнений.

% Использование кривых Безье, особенно кривых высших порядков ($N\geq4$), позволяет лучше описать подобные сложные процессы устаревания.
% Также это дает возможность описывать не только характер процесса, но и попытки повлиять на него.
% Пример использования кривой Безье для аппроксимации процесса устаревания показан на рис. \ref{fig:exp_vs_curve}.
% begin{figure}[h!]
%     \centering
%     \includegraphics[width=90mm]{exp_vs_curve.png}
%     \caption{Применение кривой Безье для описания процесса и сравнение с экспонентой}
%     \label{fig:exp_vs_curve}
% \end{figure}

% Ошибка аппроксимации определенная при помощи метода наименьших квадратов, показана в табл. \ref{tabl_3}, где можно видеть значительное преимущество аппроксимации сложной функции устаревания кривыми Безье.

% \begin{table}[h!]
%     \caption{Ошибки аппроксимации для разных функций}\vspace*{2mm}
%     \centering \small\label{tabl_3}

%     \begin{tabular}{c|c}
%         \hline
%         Тип функции & Ошибка \\
%         \hline
%         Кусочно-заданная функция  & $\epsilon=0.56$ \\
%         \hline
%         Кривая Безье & $\epsilon=0.07$             \\
%     \end{tabular}
% \end{table}

% В разделе экспериментальных исследований описано исследование наиболее опасных критических уязвимостей в программном обеспечении, используемом в облачных средах, а также эксплуатация уязвимости CVE-2016-5195 в производственной среде.
% В подразделе эксплуатации уязвимости также описана защита от уязвимости и меры по предотвращению, мониторингу и оперативному реагированию на подобные инциденты.

% \subsection{Критические уязвимости в 2016~г.}

% Критические уязвимости 2016~г. в программном обеспечении \cite{cvedetails}, используемом в облачной среде представлено в табл. \ref{vulns}.
% \begin{table}[H]
%   \caption{Наиболее опасные критические уязвимости 2016~г.}\label{vulns}
%   \begin{tabular}{|l|p{2cm}|p{7cm}|l|}
%   \hline \hyperlink{cve}{CVE} \hyperlink{id}{ID} & \hyperlink{cvss}{CVSS} & Тип уязвимости & ПО \\
%   \hline CVE-2016-5195 & 7.2 & Получение привилегий & Linux Kernel \\
%   \hline CVE-2016-6258 & 7.2 & Получение привилегий & Xen \\
%   \hline CVE-2016-5696 & 5.8 & Получение данных & Linux Kernel \\
%   \hline CVE-2016-3710 & 7.2 & Запуск кода & QEMU \\
%   \hline CVE-2016-8655 & 7.2 & DoS, получение привилегий & Linux Kernel \\
%   \hline CVE-2016-4997 & 7.2 & DoS, получение привилегий, доступ к памяти & Linux Kernel \\
%   \hline CVE-2016-4484 & 7.2 & Получение привилегий & CryptSetup \\
%   \hline CVE-2016-6309 & 10.0 & DoS, запуск кода & OpenSSL\\
%   \hline CVE-2016-1583 & 7.2 & Переполнение стека, получение привилегий, DoS & Linux Kernel \\
%   \hline
%   \end{tabular}
% \end{table}

% Рассмотрим подробнее уязвимости из этой таблицы.

% Опасная уязвимость CVE-2016-5195 <<Dirty COW>> обнаружена Филом Остером в ходе исследования зараженного сервера.
% Данная уязвимость присутствовала в составе ядра Linux 10 лет, начиная с версии 2.6.22.
% Исправлена в октябре 2016~г \cite{dcow}.

% Суть уязвимости состоит в том, что при чтении области данных памяти при использовании механизма copy-on-write (\hyperlink{cow}{COW}), используется одна общая копия, а при изменении данных --- создается новая копия, а так называемый <<Dirty Bit>> указывает был ли изменен соответствующий блок памяти.
% Проблема возникает при одновременном вызове функции madvise() и записи в страницу памяти, к которой пользователь не имеет доступа на изменение.
% При многочисленном повторении запросов происходит <<гонка>> (race condition) и эксплоит получает право на изменение страницы памяти, которая может относится к привилегированному suid-файлу.

% Таким образом, с использованием эксплоитов можно повысить привилегии локального пользователя например до пользователя root.
% В выпущенном патче добавлен соответствующий флаг FOLL\_COW, который является индикатором окончания операции COW:
% \begin{lstlisting}
% diff --git a/include/linux/mm.h b/include/linux/mm.h
% +#define FOLL_COW 0x4000  /* internal GUP flag */

% diff --git a/mm/gup.c b/mm/gup.c
% +static inline bool can_follow_write_pte(pte_t pte, unsigned int flags)
% +{
% +  return pte_write(pte) ||
% +  ((flags & FOLL_FORCE) && (flags & FOLL_COW) && pte_dirty(pte));
% +}

% -if ((flags & FOLL_WRITE) && !pte_write(pte)) {
% +if ((flags & FOLL_WRITE) && !can_follow_write_pte(pte, flags)) {

% -*flags &= ~FOLL_WRITE;
% +*flags |= FOLL_COW;
% \end{lstlisting}

% Уязвимость в гипервизоре Xen --- CVE-2016-6258 (XSA-182) позволяет привилегированному пользователю гостевой системы выполнить свой код на уровне хост-системы.
% Уязвимость применима только к режиму паравиртуализации и не работает в режиме аппаратной виртуализации, а также на архитектуре \hyperlink{arm}{ARM}.

% В паравиртуализированных окружениях, для быстрого обновления элементов в таблице страниц памяти пропускались ресурсоемкие повторные проверки доступа.
% Данные проверки реализовывались с помощью очистки Access/Dirty битов, однако этого оказалось недостаточно \cite{xsa182}.
% В патче к уязвимости представлен код, в котором исправлены проверки доступов, а именно добавлены дополнительные повторные проверки:
% \begin{lstlisting}
% diff --git a/xen/include/asm-x86/page.h b/xen/include/asm-x86/page.h
% +#define _PAGE_AVAIL_HIGH (_AC(0x7ff, U) << 12)

% diff --git a/xen/arch/x86/mm.c b/xen/arch/x86/mm.c
% +#define FASTPATH_FLAG_WHITELIST \
% +(_PAGE_NX_BIT | _PAGE_AVAIL_HIGH | _PAGE_AVAIL | _PAGE_GLOBAL | \
% +_PAGE_DIRTY | _PAGE_ACCESSED | _PAGE_USER)

% -if ( !l1e_has_changed(ol1e, nl1e, PAGE_CACHE_ATTRS | _PAGE_RW | _PAGE_PRESENT) )
% +if ( !l1e_has_changed(ol1e, nl1e, ~FASTPATH_FLAG_WHITELIST) )

% -if ( !l2e_has_changed(ol2e, nl2e, unlikely(opt_allow_superpage)
% -? _PAGE_PSE | _PAGE_RW | _PAGE_PRESENT : _PAGE_PRESENT) )
% +if ( !l2e_has_changed(ol2e, nl2e, ~FASTPATH_FLAG_WHITELIST) )

% -if ( !l3e_has_changed(ol3e, nl3e, _PAGE_PRESENT) )
% +if ( !l3e_has_changed(ol3e, nl3e, ~FASTPATH_FLAG_WHITELIST) )

% -if ( !l4e_has_changed(ol4e, nl4e, _PAGE_PRESENT) )
% +if ( !l4e_has_changed(ol4e, nl4e, ~FASTPATH_FLAG_WHITELIST) )
% \end{lstlisting}

% Уязвимость CVE-2016-5696 в ядре Linux, позволяющая вклиниться в стороннее TCP-соединение была обнародована на конференции Usenix Security Symposium.
% Из-за недоработки механизмов ограничения интенсивности обработки ACK-пакетов, существует возможность вычислить информацию о номере последовательности, которая идентифицирует поток в TCP-соединении, и со стороны отправить подставные пакеты, которые будут обработаны как часть атакуемого соединения \cite{tcp}.

% Суть атаки заключается в наводнении хоста запросами для срабатывания ограничения в обработчике ACK-пакетов, параметры которого можно получить меняя характер нагрузки.
% Для атаки необходимо создать шум, чтобы определить значение общего счетчика ограничения интенсивности ACK-ответов, после чего на основании оценки изменения числа отправленных пакетов определить номер порта клиента и осуществить подбор номера последовательности для конкретного TCP-соединения.

% В патче к этой уязвимости увеличен лимит TCP ACK и добавлена дополнительная рандомизация для снижения предсказуемости параметров работы системы ограничения ACK-пакетов:
% \begin{lstlisting}
% diff --git a/net/ipv4/tcp_input.c b/net/ipv4/tcp_input.c
% +int sysctl_tcp_challenge_ack_limit = 1000;
% -u32 now;
% +u32 count, now;
% +u32 half = (sysctl_tcp_challenge_ack_limit + 1) >> 1;

% -challenge_count = 0;
% +WRITE_ONCE(challenge_count, half +
% +prandom_u32_max(sysctl_tcp_challenge_ack_limit));

% -if (++challenge_count <= sysctl_tcp_challenge_ack_limit) {
% +  count = READ_ONCE(challenge_count);
% +if (count > 0) {
% +  WRITE_ONCE(challenge_count, count - 1);
% \end{lstlisting}

% Уязвимость CVE-2016-3710 <<Dark portal>> обнаружена в QEMU, который используется для эмуляции оборудования в Xen и KVM.
% При использовании метода эмуляции stdvga, из-за записи в регистр памяти VBE\_DISPI\_INDEX\_BANK, хранящий смещение адреса текущего банка видеопамяти, возможно обращение к областям памяти, выходящим за границы буфера, так как предлагаемые банки видеопамяти адресуются с использованием типа byte (uint8\_t *), а обрабатываются как тип word (uint32\_t *) \cite{qemu}.
% Уязвимость позволяет выполнить код на хост-система с правами обработчика QEMU (обычно root или qemu-dm).

% В патче к уязвимости добавлены дополнительные проверки диапазона памяти:
% \begin{lstlisting}
% diff --git a/hw/display/vga.c b/hw/display/vga.c
% +assert(offset + size <= s->vram_size);
% -if (s->vbe_regs[VBE_DISPI_INDEX_BPP] == 4) {
% -  val &= (s->vbe_bank_mask >> 2);
% -} else {
% -  val &= s->vbe_bank_mask;
% -}
% +val &= s->vbe_bank_mask;
% +assert(addr < s->vram_size);

% -ret = s->vram_ptr[((addr & ~1) << 1) | plane];
% +addr = ((addr & ~1) << 1) | plane;
% +if (addr >= s->vram_size) {
% +  return 0xff;
% +}
% +ret = s->vram_ptr[addr];

% +if (addr * sizeof(uint32_t) >= s->vram_size) {
% +  return 0xff;
% +}

% +assert(addr < s->vram_size);
% +if (addr >= s->vram_size) {
% +  return;
% +}

% +if (addr * sizeof(uint32_t) >= s->vram_size) {
% +  return;
% +}
% \end{lstlisting}

% Уязвимость в ядре Linux CVE-2016-8655 позволяет злоумышленнику запустить код на уровне ядра с использованием функции packet\_set\_ring() через манипуляции с кольцевым буфером TPACKET\_V3 \cite{netraw}.
% Использовать уязвимость можно для выхода за пределы контейнера, для этого необходимо чтобы локальный пользователь имел полномочия по созданию сокетов AF\_PACKET.

% Устранена уязвимость в декабре 2016~г. с помощью кода, перехватывающий возможные гонки:
% \begin{lstlisting}
% diff --git a/net/packet/af_packet.c b/net/packet/af_packet.c
% -if (po->rx_ring.pg_vec || po->tx_ring.pg_vec)
% -  return -EBUSY;

% -po->tp_version = val;
% -return 0;
% +break;

% +lock_sock(sk);
% +if (po->rx_ring.pg_vec || po->tx_ring.pg_vec) {
% +  ret = -EBUSY;
% +} else {
% +  po->tp_version = val;
% +  ret = 0;
% +}
% +release_sock(sk);
% +return ret;

% +lock_sock(sk);
% +release_sock(sk);
% \end{lstlisting}

% Уязвимость CVE-2016-4997 присутствует в подсистеме netfilter ядра Linux и связана с недоработкой в обработчике setsockopt IPT\_SO\_SET\_REPLACE и может быть использована в системах, использующих изолированные контейнеры \cite{netf}.
% Проблема проявляется при использовании пространств имен для изоляции сети и идентификаторов.

% В патче к уязвимости была добавлена дополнительная проверка адреса функции xt\_entry\_foreach():
% \begin{lstlisting}
% diff --git a/net/ipv4/netfilter/arp_tables.c b/net/ipv4/netfilter/arp_tables.c
% +if (pos + size >= newinfo->size)
% +  return 0;

% +e = (struct arpt_entry *)
% +(entry0 + newpos);

% +if (newpos >= newinfo->size)
% +  return 0;

% -if (!mark_source_chains(newinfo, repl->valid_hooks, entry0)) {
% -  duprintf("Looping hook\n");
% +if (!mark_source_chains(newinfo, repl->valid_hooks, entry0))
% -}

% diff --git a/net/ipv4/netfilter/ip_tables.c b/net/ipv4/netfilter/ip_tables.c
% +if (pos + size >= newinfo->size)
% +  return 0;

% +e = (struct ipt_entry *)
% +(entry0 + newpos);
% +if (newpos >= newinfo->size)
% +  return 0;

% diff --git  a/net/ipv6/netfilter/ip6_tables.c b/net/ipv6/netfilter/ip6_tables.c
% +if (pos + size >= newinfo->size)
% +  return 0;

% +e = (struct ip6t_entry *)
% +(entry0 + newpos);
% +if (newpos >= newinfo->size)
% +  return 0;
% \end{lstlisting}

% Уязвимость CVE-2016-4484 выявлена в пакете CryptSetup, применяемом для шифрования дисковых разделов в Linux.
% Ошибка в коде скрипта разблокировки позволяет получить доступ в командную оболочку начального загрузочного окружения с правами суперпользователя.

% Несмотря на шифрование разделов возможны ситуации при которых некоторые разделы не шифруются (например /boot), таким образом возможно оставить в системе исполняемый файл с правами setuid root для повышения привилегий или скопировать шифрованный раздел по сети для подбора пароля \cite{cryptsetup}.

% Для эксплуатации уязвимости необходимо удерживать клавишу Enter в ответ на запрос доступа к зашифрованным разделам.
% Спустя 70 секунд удерживания клавиши осуществися автоматический вход в командную оболочку.

% Проблема вызвана некорректной обработкой лимита на максимальное число попыток монтирования.
% Исправление доступно в виде загрузки \hyperlink{grub}{GRUB} с параметром <<panic>> или в виде патча, устанавливающего лимит:
% \begin{lstlisting}
% diff --git a/scripts/local-top/cryptroot b/scripts/local-top/cryptroot
% +success=0

% +  success=1

% -if [ $crypttries -gt 0 ] && [ $count -gt $crypttries ]; then
% -  message "cryptsetup: maximum number of tries exceeded for $crypttarget"
% -  return 1
% +if [ $success -eq 0 ]; then
% +  message "cryptsetup: Maximum number of tries exceeded. Please reboot."
% +  while true; do
% +    sleep 100
% +  done
% \end{lstlisting}

% Сотрудники компании Google выявили уязвимость CVE-2016-6309 в OpenSSL, позволяющая злоумышленникам выполнить произвольный код при обработке отправленных ими пакетов.
% При получении сообщения размером больше 16~Кб, при перераспределении памяти остается висячий указатель на старое положение буфера и запись поступившего сообщения производится в уже ранее освобожденную область памяти.

% В патче опубликованном в сентябре 2016~г. исправлена проблема с висячим указателем:
% \begin{lstlisting}
% diff --git a/ssl/statem/statem.c b/ssl/statem/statem.c
% +static int grow_init_buf(SSL *s, size_t size) {
% +  size_t msg_offset = (char *)s->init_msg - s->init_buf->data;
% +  if (!BUF_MEM_grow_clean(s->init_buf, (int)size))
% +    return 0;
% +  if (size < msg_offset)
% +    return 0;
% +  s->init_msg = s->init_buf->data + msg_offset;
% +  return 1;
% +}

% -&& !BUF_MEM_grow_clean(s->init_buf,
% -(int)s->s3->tmp.message_size
% -+ SSL3_HM_HEADER_LENGTH)) {
% +&& !grow_init_buf(s, s->s3->tmp.message_size
% ++ SSL3_HM_HEADER_LENGTH)) {
% \end{lstlisting}

% Ядро Linux подвержено уязвимости CVE-2016-1583, благодаря которой существует возможность поднятия привилегий локальному пользователю при помощи eCryptfs.

% С помощью формирования рекурсивных вызовов в пространстве пользователя возможно добиться переполнения стека ядра.
% Злоумышленник может организовать цепочку рекурсивных отражений в память файла, при которой процесс отражает в свое окружения другие файлы \cite{ecryptfs}.
% При чтении содержимого файлов будет вызван обработчик pagefault для процессов, что приведет к переполнению стека.

% Соответствующие исправления были приняты в июне 2016~г. в ядре Linux:
% \begin{lstlisting}
% diff --git a/fs/proc/root.c b/fs/proc/root.c
% +sb->s_stack_depth = FILESYSTEM_MAX_STACK_DEPTH;

% diff --git a/fs/ecryptfs/kthread.c b/fs/ecryptfs/kthread.c
% +#include <linux/file.h>

% -goto out;
% +goto have_file;

% -if (IS_ERR(*lower_file))
% +if (IS_ERR(*lower_file)) {
% +  goto out;
% +}
% +have_file:
% +  if ((*lower_file)->f_op->mmap == NULL) {
% +    fput(*lower_file);
% +    *lower_file = NULL;
% +    rc = -EMEDIUMTYPE;
% +  }
% \end{lstlisting}


% \subsection{Эксплуатация уязвимости ядра Linux CVE-2016-5195}

% Наиболее опасной уязвимостью 2016~г. является CVE-2016-5195 с кодовым именем <<Dirty COW>>.
% Такая степень опасности обусловлена тем, что ей подвержены практически все ядра Linux начиная с версии 2.6.22, а также легкостью применения эксплоитов.

% В данном разделе применяется эксплоит \cite{dcowexp}, позволяющий получить права суперпользователя из-под локального пользователя.

% Для эксплуатации используется дистрибутив CentOS:
% \begin{lstlisting}
% # cat /etc/redhat-release
% CentOS Linux release 7.2.1511 (Core)
% # uname -r
% 3.10.0-327.el7.x86_64
% \end{lstlisting}

% Необходимо повысить привилегии локального пользователя dcow до суперпользователя root, для этого необходимо скачать эксплоит и установить инструменты для компиляции (gcc/g++).
% После установки всех необходимых настроек компилируем код эксплоита:
% \begin{lstlisting}
% $ id
% uid=1000(dcow) gid=1000(dcow) groups=1000(dcow)
% $ git clone https://github.com/gbonacini/CVE-2016-5195
% $ cd CVE-2016-5195/
% $ make
% g++ -Wall -pedantic -O2 -std=c++11 -pthread -o dcow dcow.cpp -lutil
% \end{lstlisting}

% После компиляции исходного кода эксплоита, в текущем каталоге появляется бинарный файл, запуск которого в автоматическом режиме позволяет получить права доступа суперпользователя и сменить его пароль на <<dirtyCowFun>>:
% \begin{lstlisting}
% $ ./dcow
% Running ...
% Received su prompt (Password: )
% Root password is:   dirtyCowFun
% Enjoy! :-)
% $ su root
% Password: dirtyCowFun
% # id
% uid=0(root) gid=0(root) groups=0(root)
% \end{lstlisting}

% Для исправления уязвимости в кратчайшие сроки для ядра Linux был внесен патч, а мейнтейнеры дистрибутивов опубликовали пакеты с исправлениями.
% Для обновления дистрибутива и устранения уязвимости необходимо скачать исправленную версию ядра и произвести перезагрузку.

% В случае когда целый парк серверов не имеет возможности совершать перезагрузку необходимо использовать такие технологии как kpatch, livepatch, KernelCare.
% Подобные технологии позволяют применять патчи для ядра Linux без перезагрузки.

% Технология KernelCare является коммерческим решением и позволяет без особых проблем в оперативном порядке использовать патчи для ядра, в том числе патч для CVE-2016-5195.

% Пример эксплуатации уязвимости при использовании KernelCare:

% \begin{lstlisting}
% # /usr/bin/kcarectl --uname
% 3.10.0-327.36.3.el7.x86_64
% # /usr/bin/kcarectl --patch-info  | grep CVE-2016-5195 -A3 -B3
% kpatch-name: 3.10.0/0001-mm-remove-gup_flags-FOLL_WRITE-games-from-__get_user-327.patch
% kpatch-description: mm: remove gup_flags FOLL_WRITE games from __get_user_pages()
% kpatch-kernel: >kernel-3.10.0-327.36.2.el7
% kpatch-cve: CVE-2016-5195
% kpatch-cvss: 6.9
% kpatch-cve-url: https://access.redhat.com/security/cve/cve-2016-5195
% kpatch-patch-url: https://git.kernel.org/linus/19be0eaffa3ac7d8eb6784ad9bdbc7d67ed8e619
% $ ./dcow
% Running ...
% \end{lstlisting}

% Как видно из вывода команд, эксплоит не работает при включенном KernelCare.

% Пример отключения KernelCare и попытка эксплуатации уязвимости:
% \begin{lstlisting}
% # /usr/bin/kcarectl --unload
% Updates already downloaded
% KernelCare protection disabled, kernel might not be safe
% # su - dcow
% $ cd CVE-2016-5195/
% $ ./dcow
% Running ...
% Received su prompt (Password: )
% Root password is:   dirtyCowFun
% Enjoy! :-)
% \end{lstlisting}

% В облачной среде, где при каждой минуте простоя поставщик облачных услуг теряет деньги, использование подобных технологий позволяет существенно уменьшить время простоя серверов.

% \subsection{Мониторинг уязвимостей в программном обеспечении}

% Для своевременного реагирования на уязвимости в ПО, используемом в облачной среде необходим их мониторинг.
% В ходе исследования различных открытых баз уязвимостей и систем мониторинга не было обнаружено автоматизированных инструментов, позволяющих интегрировать поиск уязвимостей в систему мониторинга.

% Для написания такой системы использовалась открытая база уязвимостей сайта www.cvedetails.com \cite{cvedetails}.
% Сайт cvedetails позволяет в ограниченном режиме получить доступ к базе посредством \hyperlink{json}{JSON} API.

% Возможна интеграция программы с любой системой мониторинга или запуском по расписанию.
% Существует поддержка указания даты поиска уязвимостей, по умолчанию скрипт ищет уязвимости на сегодняшний день:
% \begin{lstlisting}
% $ ./vulncontrol.py -d 2017-02-18 -m 5
% CVE-2017-6074 9.3 http://www.cvedetails.com/cve/CVE-2017-6074/
% CVE-2017-6001 7.6 http://www.cvedetails.com/cve/CVE-2017-6001/
% CVE-2017-5986 7.1 http://www.cvedetails.com/cve/CVE-2017-5986/
% Telegram alert not sent
% \end{lstlisting}

% Программа получает данные от сайта в JSON-формате вида:
% \begin{lstlisting}
% {
%     "cve_id": "CVE-2017-5551",
%     "cvss_score": "3.6",
%     "cwe_id": "264",
%     "exploit_count": "0",
%     "publish_date": "2017-02-06",
%     "summary": "The simple_set_acl function in fs/posix_acl.c ...",
%     "update_date": "2017-02-09",
%     "url": "http://www.cvedetails.com/cve/CVE-2017-5551/"
% }
% \end{lstlisting}


% Поддерживается интеграция приложения с мессенджером Telegram (рис. \ref{tscreen}).

% \addimg{tscreen}{0.7}{Уведомление программы в Telegram}{tscreen}

% Описание кодов выхода программы представлено в табл. \ref{exitcodes}:
% \begin{table}[H]
%   \caption{Коды выхода программы}\label{exitcodes}
%   \begin{tabular}{|l|l|l|}
%   \hline Код & Сообщение & Отправлено в Telegram? \\
%   \hline 0 & Уязвимости не найдены & Нет \\
%   \hline 1 & Уязвимости найдены & Нет \\
%   \hline 2 & Уязвимости найдены & Да \\
%   \hline 3 & Уязвимости найдены & Нет, неверные токен и идентификатор \\
%   \hline
%   \end{tabular}
% \end{table}

Исходный код программы представлен в прил. \hyperlink{app-a}{А}.

\clearpage
