\section{АНАЛИЗ ИНФОРМАЦИОННЫХ ПРОЦЕССОВ В ЗАДАЧЕ АНАЛИЗА ПРОЦЕССА ДЕГРАДАЦИИ}

В существующих системах в области анализа и совершенствования ТС напрямую не учитывается изменение эффективности во времени в процессе жизненного цикла. 
Используемые данными системами модели основываются на утверждении об экспоненциальном законе устаревания либо закон вообще не упоминают. 

Рассмотрим подробнее предметную область анализа процесса деградации и процессы, связанные с работой СППР в рамках данной области.

\subsection{Стандарты описания процессов в рамках прикладной предметной области}
Для того чтобы построить адекватную модель деградации реальной системы следует учитывать скорость изменения процесса изменения ИПЭ системы вследствие процесса деградации, а также готовность системы к реинжинирингу. 

В общем случае изменение ИПЭ системы во времени $E(t)$ представимо в виде:

\begin{equation} \label{eq:oldModel}
    E(t) = \begin{cases}
        E_0,            & \; t<t_m         \\
        E_0 \cdot f(t), & \; t<t_m         \\
        E_0 \cdot r,          & \; t>t_r, r>1 \\
    \end{cases}
\end{equation}

% \begin{conditions*}
%          E(t)       &   эффективность ТС в заданный момент времени, \\
%          E_0       &   начальная эффективность ТС, \\
%          t_m       &   время начала морального и $\backslash$ или технического устаревания системы, \\
%          t_r       &   время начала интенсивного совершенствования (модернизации) системы, \\
%          r         &   коэффициент реинжиниринга, который может быть ограничен возможным уровнем автоматизации системы, 
%          f(t)       &  функция деградации.
% \end{conditions*}

На рисунке \ref{degradation_model} показана модель деградации комплексной системы. 
На первом отрезке $E(t) = const$. 
Это означает, что ИПЭ практически не изменяется. 
На отрезке t > tm показатель эффективности падает и определяется в общем случае некоторой функцией f(t). 
Третий отрезок характеризует период реинжиниринга, который повышает показатель эффективности по отношению к исходной эффективности системы (E0) [5].

\addimg{degradation_model}{0.6}{Модель изменения ИПЭ}{degradation_model}

Теоретический диапазон гибкости реинжиниринга иллюстрируется как площадь треугольника $ABC$. Верхняя часть треугольника может деформироваться и ограничиваться, например, экспонентой и определять фактический диапазон гибкости. При этом, изгиб экспоненты может быть различным, что оказывает существенное влияние на показатель эффективности ИС. Время принятия решения о начале реинжиниринга связано с выходом эффективности ИС на допустимую границу (заданную в документации) и определяется условием (1.2):

где	$E_{min}$ — минимальный допустимый уровень показателя эффективности системы.

В ходе анализа процесса устаревания возможно составить табличную функцию, описывающую динамику изменения эффективности ИС в процессе устаревания. 
Для упрощения процедуры анализа предлагается использовать аппроксимацию данной табличной функции.

Рассмотрим следующую классификацию технических систем в зависимости от функции деградации:

А - считать систему слабо устаревающей, если после начала морального устаревания (в период, равный 1 рабочему циклу системы) изменение
показателя ее эффективности не превышает 5-10 \%. 
На рисунке \ref{degradation_model} эта ситуация отражена функцией $E_0 \cdot e^{k(t−tm)}$;

Б - считать систему интенсивно устаревающей, если после начала устаревания (в период равный 1 рабочему циклу системы) изменение показателя ее эффективности превышает 20%;

В - считать систему замедленно устаревающей (системой высокой готовности). 

Классификация также отражена на рисунке \ref{classificationl}.

\addimg{classification}{1}{Условная классификация типов устаревания систем}{classification}
 
Разрабатываемся СППР должна давать возможность построения различные типы моделей деградации систем. 
Построение и применение адекватной модели деградации позволяет решать следующие задачи:
\begin{itemize}
  \item определение момента экономически целесообразного начала реинжиниринга;
  \item расчет эффективности ИС в заданный момент времени;
  \item расчет показателей устаревания ИС и подстройка параметров ИС под заданные показатели устаревания.
\end{itemize}

Таким образом, целесообразно определить модели устаревания ИС, основанные на различных законах распределения случайных величин, а также получить расчеты параметров эффективности системы при ее устаревании, на основе анализа зависимости этих параметров от вида закона распределения.

Проведем декомпозицию системы для структуризации целей, выявления проблем и противоречий. 
Построим функциональную модель СППР, отражающую структуру и функции системы [11,12,13]. 

Для исследования и проектирования систем на логическом уровне используется стандарт IDEF0. 
Данный стандарт часто используется для описания и оптимизации процессов в анализируемой системе. 

Построение модели IDEF0 начинается с представления всей системы в виде контекстной диаграммы.
Для построения модели использовался продукт Ramus Process Builder 1.2.5. 

Контекстная диаграмма верхнего уровня проектируемой системы представлена на рисунке \ref{idef0_context}.

\addimg{idef0_context}{1}{Контекстная диаграмма верхнего уровня}{idef0_context}

Расшифровка данной диаграммы представлена в таблице А.1.

Описание анализируемой системы обычно содержит в себе параметры системы, критерии, методику расчета эффективности. 
Также существует возможность отправки значений параметров системы автоматически, либо задания табличных значений ИПЭ во времени. 
В случае построения модели на основе табличных данных используется аппроксимация данного набора значений при помощи различных математических методов.


В качестве управляющих параметров может использоваться уже существующая модель устаревания ИС, необходимая для расчетов других параметров и принятия решения. 
Также она может быть уточнена в процессе работы системы. 
Также в вектор управляющих параметров входят и ограничения системы, если расчет ведется с целью определения параметров системы с заданной функцией устаревания.

В качестве механизма обычно используются различные вспомогательные программные средства для моделирования, например AnyLogic или Simulink, а также дополнительное ПО.

На выход процесса поступают обновленная модель устаревания, рассчитанные параметры процесса устаревания, решение о целесообразности проведение реконструкции(реинжиниринга) системы, а также параметры системы в случае их расчетов на основе заданной модели устаревания.

С помощью диаграммы декомпозиции первого уровня показана детализация процессов, протекающих при преобразовании главной функции системы. Это проиллюстрировано на рисунке \ref{idef0_decompose}.

В таблице А.2 приводится описание процессов, изображенных на данной диаграмме.

\addimg{idef0_decompose}{1}{Детализация процессов, протекающих при преобразовании главной функции системы}{idef0_decompose}

Модель системы, а также ее параметры и критерии поступают на вход процесса определения значения параметров системы. 
Результаты данных вычислений в виде вектора поступают на вход процесса расчета эффективности. 
Данный процесс использует модель деградации системы в качестве управления. 
В данном процессе происходит расчет интегрального показателя эффективности. 

В процессе уточнения модели системы определяется наиболее эффективный набор параметров, чтобы динамика изменения ИПЭ удовлетворяла заданным ограничениям. 
В качестве входных данных используются данные об изменении ИПЭ в течении заданного промежутка времени. 
Ограничения, накладываемые на задачу моделирования, используются в качестве управления. 
В качестве механизмов(ресурсов) используется вспомогательное ПО. 
Наиболее оптимальная модель деградации системы является выходным параметром.

Также полученная модель устаревания поступает в качестве входных данных в процесс поддержки принятия решений. 
В рамках данного процесса ЛПР предоставляются возможные варианты действий с системой.
Принятые решения являются выходными данными данного процесса.

\subsection{Анализ литературных источников}

Выполним обзор литературных источников, которые могут помочь в решении поставленных задач. 
Для этого необходимо выполнить научный поиск в области устаревания систем, жизненного цикла систем, применения сплайнов для описания экспериментальных данных, а также дополнительные области системотехнического анализа и разработки веб-приложений.

В работе [6] исследован жизненный цикл технических систем на примере информационной системы. 
Изучена проблема управления техническими системами на основе их ЖЦ. 
Данная работа раскрывает подробнее роль жизненного цикла при анализе комплексных систем.

В статье [5] рассмотрена проблема устаревания ИС и изменения ее эффективности вследствие этого процесса. 
Предложена модель изменения показателя эффективности ИС в виде системы уравнений в зависимости от времени. 
Также затронуто понятие функции устаревания и предложен математический аппарат для описания данной функции при помощи нелинейных функций. 
В качестве примера рассмотрены такие нелинейные функции, как логарифмическая, показательная и другие.
Установлено влияние вида закона распределения на модель анализируемой информационной системы.

В статье [7] предложены факторы, позволяющие отнести информационные системы в разряд устаревших. 
В их число включаются: недостатки архитектуры системы, изменение рабочих процессов, невозможность взаимодействия с новыми системами, отсутствие хорошо разбирающихся в системе сотрудников и невозможность нанять таковых, нестабильность работы оборудования, на котором функционирует система. 
Предложенные факторы могут стать основой для разработки более комплексной системы критериев и более комплексного подхода к оценке эффективности ИС и ее показателя деградации с течением времени.

В статье [9] обоснованы сроки и виды работ по реинжинирингу информационной системы с учетом результатов мониторинга процесса ее морального старения. 
Рассмотрены три вида морального старения: экономическое, функциональное и деградация ресурсной отказоустойчивости. 
Изложенные результаты могут быть полезны для развития методического обеспечения технического регулирования при оценке и подтверждении соответствия информационных систем на стадиях проектирования и эксплуатации [9].

В статье [9] сформированы требования к системе поддержки принятия решения для управления процессом функционирования информационной системы в рамках анализа ее жизненного цикла. 
Данная система облегчает исследование процесса деградации информационной системы и может дать рекомендацию по целесообразности проведения реинжиниринга в заданное пользователем системы время.

Однако в данной работе рассматривается исключительно моделирование данной СППР при помощи математических методов. 
Авторы не предлагают практические методы ни по реализации данной СППР, ни по управлению ЖЦ ИС в рамках процесса устаревания.

Работа [10] поясняет, что основу управления устареванием составляет совокупность запланированных и скоординированных действий, которые обеспечивают показатели надежности телекоммуникационного оборудования в соответствии с техническими условиями на протяжении всего срока его службы посредством поставок заменяемых составных частей и проведения поддерживающих мероприятий. 
Данная работа показывает другой взгляд на проблему устаревания, и связывает деградацию ТС с анализом ее надежности.

В работе [11] описываются причины и особенности устаревания технических систем. 
Анализируются тенденции и формулируются методологии на данную проблему. 
К сожалению, данная работа не предлагает каких-либо практических действий по анализу комплексных технических систем.

В работе [12] представлена методика обработки экспериментальных данных с помощью кубических сплайнов. 
На примере показана возможность нахождения максимального значения кривой и интеграла под кривой, заданной отдельными точками. 
Данная работа доказывает, что сплайны могут быть применены для анализа временных данных, частным случаем которых является динамика изменения ИПЭ системы.

Статья [13] также раскрывает способ аппроксимации и интерполяции данных при помощи сплайнов. 

Статья [14] исследует применение сплайнов для аппроксимации динамических рядов. 
Особенность таких рядов – условия для начала ряда и конца ряда не полностью определены. 
Рассматривается вариант «разгона» метода аппроксимации и интерполяции сплайнами на примере параболического полинома, редко применяемого на практике.
Данный метод может быть применен в процессе прогнозирования ИПЭ в конкретный момент времени.

Статья [15] описывает модель применения различных сглаживающих сплайнов в задачах непараметрической регрессии для создания аппроксимации различных экспериментальных данных и временных рядов. 
В этой статье обсуждаются два метода непараметрической регрессии, называемые штрафным сплайном (P-spline) и кубическим сглаживающим сплайном. 
Основной целью данной статьи является сравнение этих методов использованных для прогнозирования непараметрических регрессионных моделей. 
Для сравнения данных методов в исследовании используется набор данных ежедневного обменного курса валют -турецкой лиры/доллара США на протяжении 2005-2009 гг. 
Результаты проведенного анализа показали, что модели штрафных сплайнов (P-spline) определяют лучшую аппроксимацию, чем модели кубических сглаживающих сплайнов []. 
Данный метод может быть использован для поиска наиболее подходящих параметров критериев в рамках поиска оптимальной модели деградации, удовлетворяющей заданным ограничениям.

\subsection{Выбор средств и методов решения задачи на основе вариантного анализа}
Выберем наиболее подходящий способ реализации системы при помощи метода анализа иерархий(МАИ). 
Метод состоит в разложении проблемы на все более простые составные части и дальнейшей обработке последовательных суждений ЛПР попарным сравнениям. 
В результате может быть выражена интенсивность или относительная степень взаимодействия элементов в иерархии. 
В результате получаются численные выражения этих суждений. 
МАИ [15] включает в себя процедуры синтеза множественных суждений, получение приоритетных критериев и нахождение альтернативных решений. 
Полученные знания являются оценками в шкале отношений и соответствуют жестким оценкам. 

В качестве альтернатив используются различные архитектуры, на основе которых может быть выполнена реализация системы. 
В зависимости от выбранной архитектуры будет выбран тот или другой стек технологий, позволяющий реализовать основную функциональность системы. 

На основание дерева требований можно сформировать набор критериев, по которым будут оцениваться альтернативы

Критериями являются:
\begin{itemize}
  \item скорость работы и расчетов;
  \item управляемость системой;
  \item надежность системы и расчетов;
  \item возможность модификации и использования дополнительного ПО для моделирования.
\end{itemize}

Определим основные альтернативы – это варианты реализации архитектуры для будущей системы.
\begin{itemize}
    \item клиент-серверное приложение;
    \item приложение для рабочего стола;
    \item мобильное приложение.
\end{itemize}

Приложение для рабочего стола предполагает реализацию приложения в виде полноценного толстого клиента. 
Данный способ обладает повышенной скоростью работы, однако затрудняется поддержка и дальнейшее развитие из-за консервативности способа. 

Мобильные приложения – современный способ реализации ИС, в том числе корпоративного класса. 
Данный способ позволяет использовать систему на большом количестве устройств и в любой ситуации. 
Однако мощность таких приложений и необходимость создания сенсорного интерфейса усложняет создание интерфейса пользователя.

В результате выполнения попарных сравнений, построим матрицу парных сравнений для критериев. 
Данная матрица представлена в таблице \ref{crit_compare}.

Сравнение критериев проводилось по шкале относительной важности согласно с таблицей \ref{crit_scale}.

% \begin{table}[H]
%     \caption{Матрица парных сравнений для критериев}\label{crit_compare}
%      \begin{tabular}{|p|p|p|p|p|p|p|}
%     %{\hline} \# & $q_1$ & $q_2$ & $q_3$ & $q_4$ & $W$ & $W$ \\
%     {\hline} $q_1$ & 1 & 1 & 4 & 5 & 0.091 & 0.091 \\
%     {\hline} $q_2$ & 1 & 1 & 3 & 3 & 0.100 & 0.103 \\
%     {\hline} $q_3$ & $1/4$ & $1/3$ & 1 & 2 & 0.279 & 0.280 \\
%     {\hline} $q_4$ & $1/5$ & $1/5$ & $1/2$ & 1 & 0.526 & 0.528 \\
%     {\hline} \multicolumn{5}{|c|}{Сумма} & 0.096 & 1 \\
%     {\hline}
%     \end{tabular}
%   \end{table}

  \begin{table}[H]
    \caption{Шкала оценок критериев}\label{crit_scale}
    \begin{tabular}{|p|p|}
    {\hline} Интенсивность относительной важности & Определение \\
    {\hline} 1 & если элементы $$A_i$ и $A_k$ одинаково важны \\
    {\hline} 3 & если элемент $$A_i$ незначительно важнее элемента $A_k$ \\
    {\hline} 5 & если элемент $$A_i$ значительно важнее элемента $A_k$ \\
    {\hline} 7 & если элемент $$A_i$ явно важнее элемента $A_k$ \\
    {\hline} 9 & если элемент $$A_i$ абсолютно превосходит элемент $A_k$ \\
    {\hline} 2,4,6,8 & Используются для облегчения компромиссов между оценками, слегка отличающимися от основных чисел \\
    {\hline}
    \end{tabular}
  \end{table}

На основе данной матрицы можно легко определить значения $w_i$, а также нормированные значения $W_n$:

\begin{equation}
    W=(0.09;0.1;0.28;0.53) \\
    W_n=(0.0912;0.1003;0.2801.28;0.5283)
\end{equation}

Построим для альтернатив матрицы парных соответствий по каждому из критериев. 
Для построения матриц также используется таблица \ref{crit_scale}. 
Далее рассчитаем значения функции принадлежности.
Матрицы парных сравнений представлены в таблицах \ref{crit_q1}-\ref{crit_q4}.

% \begin{table}[H]
%     \caption{Матрица парных сравнений для альтернатив по критерию q_1}\label{crit_q1}
%     \begin{tabular}{|p|p|}
%     %{\hline} \# & $v_1$ & $v_2$ & $v_3$ & $\mu$ & $\mu$ & $W_{\mu}$ \\
%     {\hline} $v_1$ & 1 & $1/4$ & 5 & 0.160 & 0.162 & 0.847 \\
%     {\hline} $v_2$ & 4 & 1 & 7 &  0.083 & 0.084 & 0.798\\
%     {\hline} $v_3$ & $1/5$ & $1/7$ & 1 & 0.745 & 0.754 & 0.975 \\
%     {\hline} \multicolumn{4}{|c|}{Сумма} & 0.979 & \multicolumn{2}{|c|}{1} \\
%     {\hline}
%     \end{tabular}
% \end{table}

% \begin{table}[H]
%     \caption{Матрица парных сравнений для альтернатив по критерию q_2}\label{crit_q2}
%     \begin{tabular}{|p|p|}
%     %{\hline} \# & $v_1$ & $v_2$ & $v_3$ & $\mu$ & $\mu$ & $W_{\mu}$ \\
%     {\hline} $v_1$ & 1 & $1/4$ & 5 & 0.160 & 0.162 & 0.847 \\
%     {\hline} $v_2$ & 4 & 1 & 7 &  0.083 & 0.084 & 0.798\\
%     {\hline} $v_3$ & $1/5$ & $1/7$ & 1 & 0.745 & 0.754 & 0.975 \\
%     {\hline} \multicolumn{4}{|c|}{Сумма} & 0.979 & \multicolumn{2}{|c|}{1} \\
%     {\hline}
%     \end{tabular}
% \end{table}

% \begin{table}[H]
%     \caption{Матрица парных сравнений для альтернатив по критерию q_3}\label{crit_q3}
%     \begin{tabular}{|p|p|}
%    % {\hline} \# & $v_1$ & $v_2$ & $v_3$ & $\mu$ & $\mu$ & $W_{\mu}$ \\
%     {\hline} $v_1$ & 1 & $1/4$ & 5 & 0.160 & 0.162 & 0.847 \\
%     {\hline} $v_2$ & 4 & 1 & 7 &  0.083 & 0.084 & 0.798\\
%     {\hline} $v_3$ & $1/5$ & $1/7$ & 1 & 0.745 & 0.754 & 0.975 \\
%     {\hline} \multicolumn{4}{|c|}{Сумма} & 0.979 & \multicolumn{2}{|c|}{1} \\
%     {\hline}
%     \end{tabular}
% \end{table}

% \begin{table}[H]
%     \caption{Матрица парных сравнений для альтернатив по критерию q_4}\label{crit_q4}
%     \begin{tabular}{|p|p|}
%     %{\hline} \# & $v_1$ & $v_2$ & $v_3$ & $\mu$ & $\mu$ & $W_{\mu}$ \\
%     {\hline} $v_1$ & 1 & $1/4$ & 5 & 0.160 & 0.162 & 0.847 \\
%     {\hline} $v_2$ & 4 & 1 & 7 &  0.083 & 0.084 & 0.798\\
%     {\hline} $v_3$ & $1/5$ & $1/7$ & 1 & 0.745 & 0.754 & 0.975 \\
%     {\hline} \multicolumn{4}{|c|}{Сумма} & 0.979 & \multicolumn{2}{|c|}{1} \\
%     {\hline}
%     \end{tabular}
% \end{table}

Исходя из полученных матриц, определим значения нечетких множеств:

\begin{equation}
    q^1_j=(0.847;0.798;0.974) \\
    q^2_j=(0.775;0.821;0.975) \\
    q^3_j=(0.801;0.510;0.801) \\
    q^4_j=(0.793;0.407;0.395) \\
\end{equation}

Рассчитаем множество $D$ с учетом неравновесных критериев:

\begin{equation}
    D=(\frac{0.775}{v_1};\frac{0.407}{v_2};\frac{0.395}{v_3})
\end{equation}

Таким образом, наиболее подходящей альтернативой будет альтернатива $v_1$.

Для проведения экспертизы были приглашены эксперты в количестве 4 человек. Трое из них связанны с тематикой поставленной проблемы, а один нет.
Проанализируем мнения экспертов по поводу 3 альтернатив. Таблица расстановки приоритетов показана в таблице 1.7. 

\subsection{Постановка задачи}
На основе данной модели устаревания возможно создание системы поддержки принятия решений, которая позволит лицу, принимающему решения(ЛПР), решать следующие задачи:
\begin{itemize}
    \item определение сроков выхода анализируемой системы из эксплуатации;
    \item определение наиболее экономически целесообразного момента начала реинжиниринга;
    \item определение и подбор параметров анализируемой системы, необходимых для достижения заданной модели устаревания.
\end{itemize}

Определение сроков выхода анализируемой системы из эксплуатации позволяет рассчитать момент времени, когда эффективность системы будет меньше минимально приемлемой эффективности. 
После этого момента применение системы будет являться нецелесообразным. Определение этого момента позволяет ЛПР принять управленческие решения, необходимые для решения данной ситуации.

С расчетом момента времени выхода системы из эксплуатации тесно связано определение наиболее экономически целесообразного момента начала модернизации системы. 
Модернизация часто связана как с частичной, так и с полной реконструкцией системы или созданем новой системы, что влечет за собой значительные финансовые расходы. 
Расчет оптимального момента позволяет снизить данные расходы.

Также возможна обратная задача, когда ЛПР на основе заданной модели устаревания хочет определить, каким образом будут изменяться параметры модели с течением времени. 
Данная задача также называется задачей запланированного устаревания [6].

Таким образом, необходимо определить оптимальную модель системы поддержки принятия решений при анализе процесса устаревания ИС средствами функционального моделирования и при помощи формализованного описания системы в целом. 
Необходимо разработать систему поддержки принятия решений, которая позволит решить поставленные задачи. 
Данная система должна быть разработана с применением современных Web-технологий, поддерживать многопользовательский режим работы.

\subsection{Формализованная постановка задачи}
Система поддержки принятия решения при анализе процесса деградации комплексной системы в рамках анализа ее ЖЦ разрабатывается с целью повышения эффективности данного анализа. 
Определим требования к системе. Ниже приведена классификация требований.
Нефункциональные требования:
\begin{itemize}
    \item масштабируемость – возможность одновременной работы нескольких пользователей и работа сразу с несколькими системами;
    \item простота пользования системой; 
    \item корректность;
    \item надежность;
    \item наглядность;
    \item возможность гибкой настройки системы.
\end{itemize}
Функциональные требования:
\begin{itemize}
    \item построение моделей устаревания ИС, основанных на различных законах распределения случайных величин, а также с использованием визуализации на основе сплайнов;
    \item расчет времени начала максимально отсроченной модернизации;
    \item определение эффективности ИС в заданный момент времени с учетом процесса деградации;
    \item расчет экономически оптимального срока эксплуатации ИС до момента необходимости проведения модернизации;
    \item возможность изменения структуры системы(добавление/удаление блоков);
    \item расчет эффективности с использованием особенностей структуры системы;
    \item система должна выполнять вычисление коэффициента отражающего темп морального старения системы;
    \item возможность многопользовательского взаимодействия с системой;
    \item возможность одновременной работы с несколькими системами.
\end{itemize}

Обладая всеми этими возможностями, система позволит упростить исследование процесса совершенствования ТС и в частности дать рекомендацию о целесообразности модернизации ТС в заданный момент времени.

пределим подходящую модель качества путем сравнения различных моделей качества. 
Итоговая сравнительная таблица представлена в таблице 3.1:

Таким образом, наиболее оптимальной моделью качества, подходящей под заданные нефункциональные требования, является модель Мак-Кола.

Выполним анализ требований к системе, указанных выше. 
Из описания системы, определено множество функций СППР.

Глобальная функция: $Ф$ – получение комплексного проекта в виде системы поддержки принятия решения для управления процессом функционирования как всей ИС, так и отдельных ее подсистем в рамках анализа ее жизненного цикла и процесса деградации ИС. 

Поддерживающие и обеспечивающие функции:
\begin{itemize}
    \item $Ф_1$ - построение модели устаревания системы;
    \item $Ф_2$ - определение эффективности системы;
    \item $Ф_3$ - поддержка принятия решения о реинжиниринге;
    \item $Ф_4$ - поддержка анализа нескольких систем;
\end{itemize}

$Ф_1$ можно считать базовой функцией, разрабатываемой ИС, поскольку исходя из $Ф_1$ будут обеспечиваться остальные функции.

Определение адекватной модели деградации системы позволяет на основе описания системы, ее структуры, ограничений, а также способа расчета эффективности определить наиболее подходящую функцию устаревания. 
При анализе уже предоставленной статистики изменения эффективности ТС может использоваться аппроксимация с использованием сплайнов.

Функция определения эффективности системы зависит от функции определения модели, так как подразумевается использование данной модели для расчета и прогнозирования изменений эффективности КС в будущие моменты времени.

Функция поддержки принятия решений о реинжиниринге также использует модель деградации, чтобы определить наиболее экономически целесообразный момент начала реинжиниринга. 
Экономическая целесообразность играет весомую роль, т.к. достаточно часто процесс реинжиниринга сопряжен с полной переработкой системы с нуля.

Поддержка анализа нескольких систем является важной функцией, что позволит пользователю(ЛПР) сравнивать процессы устаревания, происходящие в нескольких системах, а также организовывать совместную работу в ходе анализа.

Определим поле требований как вектор требований, их шкал и ограничений: 
















Поступающие на вход данные об анализируемой системе, в системе проходят предварительную обработку. 
Далее формируется модель устаревания системы, на основе которой рассчитываются параметры устаревания, прогнозируется значение эффективности системы и формируется решение о начале процесса реинжиниринга. 
На выход поступают результаты анализа и исследований. 
Данные результаты, совместно с предложенным системой решением, используются ЛПР для принятия окончательного решения.

Можно сделать вывод, что в данном разделе был проведен анализ предметной области. 
Были определены основные математические модели процесса деградации системы. 
Построена модель процессов в СППР с использованием методологии IDEF0. 
Сформировано формализованное описание системы в виде совокупности элементов, ограничений и требований. 
В результате чего был определен закон функционирования системы.

% Конечной целью выпускной квалификационной работы магистра на тему <<Исследование процессов обеспечения безопасности облачных сред>> является повышение эффективности процессов обеспечения информационной безопасности облачных сред.
% Для достижения поставленной цели необходимо разработать структуру системы обеспечения информационной безопасности заданной облачной среды, провести внедрение и экспериментальные исследования.

% Для достижения целей исследования должны быть решены следующие задачи:
% \begin{itemize}
%   \item обзор составных частей облачной инфраструктуры;
%   \item анализ технологий используемых облачными провайдерами, необходимых для построения облачной инфраструктуры;
%   \item специфика применений облачных вычислений в России;
%   \item исследование проблемы безопасности облачных вычислений;
%   \item исследование уязвимостей в облачной среде, с использованием программ, распространяющихся под свободными лицензиями, например \hyperlink{gnu}{GNU} \hyperlink{gpl}{GPL}.
% \end{itemize}

% Входными данными для исследования являются облачная инфраструктура состоящая из физических и виртуальных серверов на базе операционной системы Linux.

% Выходными данными являются структура системы информационной безопасности и программное обеспечение для получения данных о уязвимостях в программном обеспечении, используемом в инфраструктуре, а также собранная и проанализированная информация о проблемах безопасности в облачной среде.

% Для применения практических навыков исследования уязвимостей необходима аппаратная платформа со следующими характеристиками:
% \begin{itemize}
%   \item процессор с поддержкой аппаратной виртуализации;
%   \item минимальный объем \hyperlink{ram}{ОЗУ} 8 Гб, рекомендуемый --- не менее 16 Гб;
%   \item минимум 15 Гб места на жестком диске (\hyperlink{ssd}{SSD});
%   \item операционная система Ubuntu 16.04, CentOS 7 или Debian 8 GNU/Linux.
% \end{itemize}

% Данная задача также рассматривается с точки зрения системного и вариантного анализа.

% Системный анализ включает в себя \cite{sys-analyz}:
% \begin{itemize}
%   \item системотехническое представление системы безопасности в виде <<черного ящика>>;
%   \item описание входных и выходных данных;
%   \item список функций, которые выполняет система безопасности;
%   \item учет случайностей;
%   \item декомпозицию системы и описание связей между ее элементами.
% \end{itemize}

% Вариантный анализ произведен исходя из выбранных критериев \cite{var-analyz}:
% \begin{itemize}
%   \item цена;
%   \item масштабируемость;
%   \item отказоустойчивость;
%   \item интерфейсы управления.
% \end{itemize}

\clearpage
