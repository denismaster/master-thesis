\section{АНАЛИЗ ИНФОРМАЦИОННЫХ ПРОЦЕССОВ В ЗАДАЧЕ ИССЛЕДОВАНИЯ ДЕГРАДАЦИОННЫХ ПРОЦЕССОВ СЛОЖНЫХ ТЕХНИЧЕСКИХ ОБЪЕКТОВ}
 
В существующих системах в области анализа и совершенствования ТС напрямую не учитывается изменение эффективности во времени в процессе жизненного цикла. 
Используемые данными системами модели основываются на утверждении об экспоненциальном законе устаревания либо закон вообще не упоминают. 

Рассмотрим подробнее предметную область анализа процесса деградации и процессы, связанные с работой СППР в рамках данной области.

\subsection{Процесс деградации в жизненном цикле СТО}
Согласно стандарту ISO/IEC 15288, сложные технические объекты и комплексные системы обладают жизненным циклом(ЖЦ).
Под данным термином обычно подразумевается эволюция сложных технических объектов и систем.
Данный процесс многоступенчатый и включает в себя следующие стадии \cite{Dagabyan}:
\begin{itemize}
    \item замысел;
    \item разработка;
    \item производство;
    \item эксплуатация;
    \item списание.
\end{itemize}
Уточненная модель жизненного цикла представлена на рисунке \ref{lifecycle1}.

\addimg{lifecycle1}{0.8}{Модель жизненного цикла системы}{lifecycle1}

В информационных системах(ИС), являющихся подмножеством технических систем(ТС), данное множество стадий присутствует.
Важным отличием является итеративность процесса анализа и разработки \cite{Dagabyan}.

На основе анализа жизненного цикла СТО можно эффективно управлять их работой на всех стадиях.
Согласно \cite{Doronina}, жизненный цикл играет важную роль в развитии сложных технических объектов.

Одной из важнейших стадий жизненного цикла является стадия эксплуатации.
На протяжении данной стадии СТО не выполняется разработка или производство,
но требуется поддержка инженеров. 
На этапе эксплуатации у любых технических объектов возникают деградационные процессы \cite{Miroshnikova}, которые влияют на работоспособность объектов, снижают их качества и могут привести к сбоям в работе.

В современной литературе, а также стандартах, например ГОСТ ИСО/МЭК 9126-93 \cite{ISO} важность процесса деградации СТО упомянута, но не имеет подробного описания.
Для программного обеспечения(ПО) понятия устаревания или износа практически не применяется.
Ограничения надежности и эффективности возникают из-за изменений требований к объекту, условий эксплуатации, изменения реализации объекта.
В статье \cite{Factors} предложены факторы, позволяющие отнести СТО в разряд устаревших. 
В их число включаются \cite{Serpokrylov}: недостатки архитектуры системы, изменение рабочих процессов, невозможность взаимодействия с новыми системами, 
отсутствие хорошо разбирающихся в системе сотрудников и невозможность нанять таковых, нестабильность работы оборудования, в составе которого работает объект. 

Рассмотрим следующую классификацию технических систем в зависимости от функции деградации\cite{Doronina,Serpokrylov}:

А - считать систему слабо устаревающей, если после начала морального устаревания (в период, равный 1 рабочему циклу системы) изменение
показателя ее эффективности не превышает $5-10\%$. 

Б - считать систему интенсивно устаревающей, если после начала устаревания (в период равный 1 рабочему циклу системы) изменение показателя ее эффективности превышает $20\%$;

В - считать систему замедленно устаревающей (системой высокой готовности). 

Классификация также отражена на рисунке \ref{classification}.

\addimg{classification}{0.6}{Условная классификация типов устаревания систем}{classification}

В виду неоднородности структуры СТО и наличия нескольких критериев, по которым производится оценка эффективности, процесс деградации выражается сложными зависимостями.
Примером сложного процесса устаревания может служить кривая Хайпа \cite{Litvak}, представленная на рисунке \ref{hype}

\addimg{hype}{0.6}{Пример сложного процесса деградации}{hype}

В статье \cite{Doronina} предложена модель изменения показателя эффективности СТО в виде системы уравнений в зависимости от времени. 
Также предложен математический аппарат для описания данной функции при помощи нелинейных функций. 
В качестве примера рассмотрены такие нелинейные функции, как логарифмическая, показательная и другие.

\subsection{Стандарты описания процессов в рамках прикладной предметной области}

Рассмотрим подробнее процесс жизненного цикла. 
Для исследования и проектирования на логическом уровне используется стандарт IDEF0. 
Данный стандарт часто используется для описания и оптимизации процессов в анализируемой системе. 

Построение модели IDEF0 начинается с представления всей системы в виде контекстной диаграммы.
Для построения модели использовался продукт Ramus Process Builder 1.2.5. 
Выполним построение процесса жизненного цикла СТО по методологии IDEF0. 

Контекстная диаграмма процесса жизненного цикла системы представлена на рисунке \ref{lifecycle_context}.
\addimgrotated{lifecycle_context}{1}{Контекстная диграмма процесса ЖЦ системы}{lifecycle_context}

В качестве входных данных показан замысел системы \cite{Dagabyan}. В качестве выходных данных процесса - система на этапе вывода из эксплуатации (списания).
В качестве управляющих сигналов используются стандарты и законодательные акты соответствующей предметной области СТО, а в качестве механизма - ресурсы(финансовые, людские и др.), необходимые для разработки и производства объекта.

Декомпозиция контекстной диаграммы показана на рисунке \ref{lifecycle_decomposition}
\addimgrotated{lifecycle_decomposition}{1}{Декомпозиция контекстной диаграммы процесса ЖЦ системы}{lifecycle_decomposition}.

На стадии эксплуатации возникают процессы устаревания (деградации) \cite{Miroshnikova, Serpokrylov}, что через некоторое время приводит СТО на этап списания, или вывода из эксплуатации.
Важность анализа деградационных процессов заключается в возможности управления эффективностью СТО в ходе эксплуатации.
Процесс списания может привести к финансовым и другим затратам, тогда как сложный технический объект может быть пригоден для ремонта или модернизации.

Процесс модернизации (ремонта, капитального ремонта, реинжиниринга) позволяект снизить влияние процессов устаревания, 
повысить эффективность СТО. В ходе данного процесса сложные технические объекты могут быть отремонтированы, изменены или реконструированы для работы в изменившихся условиях.
Процесс модернизации должен выполняться до того, как СТО снизит эффективность до критического уровня.
Таким образом, существует проблема определения оптимального момента модернизации системы.

\subsection{Модель процесса деградации СТО}
Для того чтобы построить адекватную модель деградации реальной системы следует учитывать скорость изменения процесса изменения ИПЭ системы вследствие процесса деградации, а также готовность системы к реинжинирингу. 

В общем случае изменение ИПЭ системы во времени $E(t)$ представимо в виде \cite{Doronina}:

\begin{equation} \label{eq:oldModel}
    E(t) = \begin{cases}
        E_0,            & \; t<t_m         \\
        E_0 \cdot f(t), & \; t<t_m         \\
        E_0 \cdot r,          & \; t>t_r, r>1 \\
    \end{cases}
\end{equation}
где $E(t)$ --- эффективность СТО в заданный момент времени,
$E_0$ --- начальная эффективность ТС, 
$t_m$ --- время начала морального и $\backslash$ или технического устаревания системы,
$t_r$ --- время начала интенсивного совершенствования (модернизации) системы,
$r$ --- коэффициент реинжиниринга, который может быть ограничен возможным уровнем автоматизации системы,
$f(t)$ --- функция деградации.

На рисунке \ref{degradation_model} показана модель деградации комплексной системы. 
На первом отрезке $E(t) = const$. 
Это означает, что ИПЭ практически не изменяется. 
На отрезке $t > t_m$ показатель эффективности падает и определяется в общем случае некоторой функцией $f(t)$. 
Третий отрезок характеризует период реинжиниринга, который повышает показатель эффективности по отношению к исходной эффективности системы ($E_0$) \cite{Doronina}.

\addimg{degradation_model}{0.6}{Модель изменения ИПЭ}{degradation_model}

Теоретический диапазон гибкости реинжиниринга иллюстрируется как площадь треугольника $ABC$. 
Верхняя часть треугольника может деформироваться и ограничиваться, например, экспонентой и определять фактический диапазон гибкости. 
При этом, изгиб экспоненты может быть различным, что оказывает существенное влияние на показатель эффективности ТС. 
Время принятия решения о начале реинжиниринга связано с выходом эффективности ТС на допустимую границу (заданную в документации) и определяется условием \ref{eq:oldModel}:

\begin{equation} \label{eq:oldModel}
    E_0(t) \leq E_{min}
\end{equation}
где	$E_{min}$ --- минимальный допустимый уровень показателя эффективности системы.

В ходе анализа процесса устаревания возможно составить табличную функцию, описывающую динамику изменения эффективности ТС в процессе устаревания. 
Для упрощения процедуры анализа предлагается использовать аппроксимацию данной табличной функции.
 
Разрабатываемся СППР должна давать возможность построения различные типы моделей деградации систем. 
Построение и применение адекватной модели деградации позволяет решать следующие задачи:
\begin{itemize}
  \item определение момента экономически целесообразного начала реинжиниринга;
  \item расчет эффективности ТС в заданный момент времени;
  \item расчет показателей устаревания ТС и подстройка параметров ТС под заданные показатели устаревания.
\end{itemize}

Таким образом, целесообразно определить модели устаревания ТС, основанные на различных законах распределения случайных величин, а также получить расчеты параметров эффективности системы при ее устаревании, на основе анализа зависимости этих параметров от вида закона распределения.

Проведем декомпозицию системы для структуризации целей, выявления проблем и противоречий. 
Построим функциональную модель СППР, отражающую структуру и функции системы \cite{Doronina, Litvak}. 

\subsection{Декомпозиция системы для структуризации цели}

Контекстная диаграмма верхнего уровня проектируемой системы представлена на рисунке \ref{idef0_context}.

\addimgrotated{idef0_context}{0.75}{Контекстная диаграмма верхнего уровня}{idef0_context}

Расшифровка данной диаграммы представлена в таблице А.1.

Описание анализируемой системы обычно содержит в себе параметры системы, критерии, методику расчета эффективности. 
Также существует возможность отправки значений параметров системы автоматически, либо задания табличных значений ИПЭ во времени. 
В случае построения модели на основе табличных данных используется аппроксимация данного набора значений при помощи различных математических методов.

В качестве управляющих параметров может использоваться уже существующая модель устаревания ТС, необходимая для расчетов других параметров и принятия решения. 
Также она может быть уточнена в процессе работы системы. 
Также в вектор управляющих параметров входят и ограничения системы, если расчет ведется с целью определения параметров системы с заданной функцией устаревания.

В качестве механизма обычно используются различные вспомогательные программные средства для моделирования, например AnyLogic или Simulink, а также дополнительное ПО.

На выход процесса поступают обновленная модель устаревания, рассчитанные параметры процесса устаревания, решение о целесообразности проведение реконструкции(реинжиниринга) системы, а также параметры системы в случае их расчетов на основе заданной модели устаревания.

С помощью диаграммы декомпозиции первого уровня показана детализация процессов, протекающих при преобразовании главной функции системы. Это проиллюстрировано на рисунке \ref{idef0_decompose}.

В таблице А.2 приводится описание процессов, изображенных на данной диаграмме.

\addimgrotated{idef0_decompose}{0.75}{Детализация процессов, протекающих при преобразовании главной функции системы}{idef0_decompose}

Модель системы, а также ее параметры и критерии поступают на вход процесса определения значения параметров системы. 
Результаты данных вычислений в виде вектора поступают на вход процесса расчета эффективности. 
Данный процесс использует модель деградации системы в качестве управления. 
В данном процессе происходит расчет интегрального показателя эффективности. 

В процессе уточнения модели системы определяется наиболее эффективный набор параметров, чтобы динамика изменения ИПЭ удовлетворяла заданным ограничениям. 
В качестве входных данных используются данные об изменении ИПЭ в течении заданного промежутка времени. 
Ограничения, накладываемые на задачу моделирования, используются в качестве управления. 
В качестве механизмов(ресурсов) используется вспомогательное ПО. 
Наиболее оптимальная модель деградации системы является выходным параметром.

Также полученная модель устаревания поступает в качестве входных данных в процесс поддержки принятия решений. 
В рамках данного процесса ЛПР предоставляются возможные варианты действий с системой.
Принятые решения являются выходными данными данного процесса.

\subsection{Анализ проблем и формализованная постановка задачи}
На основе данной модели устаревания возможно создание системы поддержки принятия решений, которая позволит лицу, принимающему решения(ЛПР), решать следующие задачи:
\begin{itemize}
    \item определение сроков выхода анализируемой системы из эксплуатации;
    \item определение наиболее экономически целесообразного момента начала реинжиниринга;
    \item определение и подбор параметров анализируемой системы, необходимых для достижения заданной модели устаревания.
\end{itemize}

Определение сроков выхода анализируемой системы из эксплуатации позволяет рассчитать момент времени, когда эффективность системы будет меньше минимально приемлемой эффективности. 
После этого момента применение системы будет являться нецелесообразным. Определение этого момента позволяет ЛПР принять управленческие решения, необходимые для решения данной ситуации.

С расчетом момента времени выхода системы из эксплуатации тесно связано определение наиболее экономически целесообразного момента начала модернизации системы. 
Модернизация часто связана как с частичной, так и с полной реконструкцией системы или созданем новой системы, что влечет за собой значительные финансовые расходы. 
Расчет оптимального момента позволяет снизить данные расходы.

Также возможна обратная задача, когда ЛПР на основе заданной модели устаревания хочет определить, каким образом будут изменяться параметры модели с течением времени. 
Данная задача также называется задачей запланированного устаревания \cite{Serpokrylov}.

Таким образом, необходимо определить оптимальную модель системы поддержки принятия решений при анализе процесса устаревания ТС средствами функционального моделирования и при помощи формализованного описания системы в целом. 
Необходимо разработать систему поддержки принятия решений, которая позволит решить поставленные задачи. 
Данная система должна быть разработана с применением современных Web-технологий, поддерживать многопользовательский режим работы.

Система поддержки принятия решения при анализе процесса деградации комплексной системы в рамках анализа ее ЖЦ разрабатывается с целью повышения эффективности данного анализа. 
Определим требования к системе. Ниже приведена классификация требований.
Нефункциональные требования:
\begin{itemize}
    \item масштабируемость – возможность одновременной работы нескольких пользователей и работа сразу с несколькими системами;
    \item простота пользования системой; 
    \item корректность;
    \item надежность;
    \item наглядность;
    \item возможность гибкой настройки системы.
\end{itemize}
Функциональные требования:
\begin{itemize}
    \item построение моделей устаревания ТС, основанных на различных законах распределения случайных величин, а также с использованием визуализации на основе сплайнов;
    \item расчет времени начала максимально отсроченной модернизации;
    \item определение эффективности ТС в заданный момент времени с учетом процесса деградации;
    \item расчет экономически оптимального срока эксплуатации ТС до момента необходимости проведения модернизации;
    \item возможность изменения структуры системы(добавление/удаление блоков);
    \item расчет эффективности с использованием особенностей структуры системы;
    \item система должна выполнять вычисление коэффициента отражающего темп морального старения системы;
    \item возможность многопользовательского взаимодействия с системой;
    \item возможность одновременной работы с несколькими системами.
\end{itemize}

Обладая всеми этими возможностями, система позволит упростить исследование процесса совершенствования ТС и в частности дать рекомендацию о целесообразности модернизации ТС в заданный момент времени.

пределим подходящую модель качества путем сравнения различных моделей качества. 
Итоговая сравнительная таблица представлена в таблице \ref{table:quality_models}:

\begin{table}[H]
    \centering
    \caption{Сравнительная таблица различных моделей качества}\label{table:quality_models}
    \begin{tabular}{|c|c|c|c|c|c|c|c|c|c|}
    \hline Характеристика качества 
    & \rotatebox[origin=c]{90}{МакКол} 
    & \rotatebox[origin=c]{90}{Боэм} 
    & \rotatebox[origin=c]{90}{FURPS} 
    & \rotatebox[origin=c]{90}{Геши} 
    & \rotatebox[origin=c]{90}{Дроми} 
    & \rotatebox[origin=c]{90}{ISO-9126} 
    & \rotatebox[origin=c]{90}{Казман}
    & \rotatebox[origin=c]{90}{Хосравн}
    & \rotatebox[origin=c]{90}{Шармоа} \\
    \hline 
    Надежность & + & + & + & + & + & + & + & & + \\
    \hline
    Корректность & + &  &  & + &  &  &  & &  \\
    \hline 
    Эффективность & + & + & + & + & + & + & + & & + \\
    \hline
    Гибкость & + &  &  & + &  &  & + & + &  \\
    \hline 
    Сопровождаемость & + & + & + & + & + & + & + & & + \\
    \hline 
    Модифицируемость &  & + &  &  &  &  &  & &  \\
    \hline 
    Возможность многократного использования & + &  &  & + &  &  &  & + & \\
    \hline 
    Устойчивость &  &  &  &  &  &  &  & + &  \\
    \hline 
    Масштабируемость &  &  &  &  &  &  &  & + &  \\
    \hline 
    Тестируемость & + & + &  &  &  &  & + &  &  \\
    \hline 
    Понятность &  & + & + &  &  &  &  &  &  \\
    \hline 
    Практичность & + &  & + & + & + & + & + & + &  \\
    \hline 
    Итого & 8 & 6 & 5 & 7 & 4 & 4 & 6 & 5 & 4 \\
    \hline
    \end{tabular}
\end{table}

Таким образом, наиболее оптимальной моделью качества, подходящей под заданные нефункциональные требования, является модель Мак-Кола.

Выполним анализ требований к системе, указанных выше. 
Из описания системы, определено множество функций СППР.

Глобальная функция: $\Phi$ – получение комплексного проекта в виде системы поддержки принятия решения для управления процессом функционирования как всей ТС, так и отдельных ее подсистем в рамках анализа ее жизненного цикла и процесса деградации ТС. 

Поддерживающие и обеспечивающие функции:
\begin{itemize}
    \item $\Phi_1$ - построение модели устаревания системы;
    \item $\Phi_2$ - определение эффективности системы;
    \item $\Phi_3$ - поддержка принятия решения о реинжиниринге;
    \item $\Phi_4$ - поддержка анализа нескольких систем;
\end{itemize}

Таким образом, формальное описание глобальной функции представлено в формуле \ref{eq:formal_global}:

\begin{equation}
    \label{eq:formal_global}
    \Phi: \left\{ \Phi_1, \Phi_2,\Phi_3,\Phi_4\ \right\}
\end{equation}
где $\Phi_1 \cdots \Phi_4 $ --- функции системы.

$\Phi_1$ можно считать базовой функцией, разрабатываемой СППР, поскольку исходя из $\Phi_1$ будут обеспечиваться остальные функции.

Определение адекватной модели деградации системы позволяет на основе описания системы, ее структуры, ограничений, а также способа расчета эффективности определить наиболее подходящую функцию устаревания. 
При анализе уже предоставленной статистики изменения эффективности ТС может использоваться аппроксимация с использованием сплайнов.

Функция определения эффективности системы зависит от функции определения модели, так как подразумевается использование данной модели для расчета и прогнозирования изменений эффективности СТО в будущие моменты времени.

Функция поддержки принятия решений о реинжиниринге также использует модель деградации, чтобы определить наиболее экономически целесообразный момент начала реинжиниринга. 
Экономическая целесообразность играет весомую роль, т.к. достаточно часто процесс реинжиниринга сопряжен с полной переработкой системы с нуля.

Поддержка анализа нескольких систем является важной функцией, что позволит пользователю(ЛПР) сравнивать процессы устаревания, происходящие в нескольких системах, а также организовывать совместную работу в ходе анализа.

Определим поле требований как вектор требований, их шкал и ограничений: 

\begin{equation}
    \label{eq:formal_requirements_1}
    T^{\Phi_1}_1=[\tau^{\Phi_1}_{11},\tau^{\Phi_1}_{12}]
\end{equation}
где $T^{\Phi_1}_1$ --- требования класса $\Phi_1$,
$\tau^{\Phi_1}_{11}$ --- адекватность построения устаревания ТС,
$\tau^{\Phi_1}_{12}$ --- тип модели деградации ТС.

Ограничения --- $\tau^{\Phi_1}_{11} = \left\{ 0,1 \right\}$,
$\tau^{\Phi_1}_{11} = $[слабое устаревания, замедленное, интенсивное устаревание, сложная модель].


\begin{equation}
    \label{eq:formal_requirements_2}
    T^{\Phi_2}_1=[\tau^{\Phi_2}_{11},\tau^{\Phi_2}_{12},\tau^{\Phi_2}_{13}]
\end{equation}
где $T^{\Phi_2}_1$ --- требования класса $\Phi_2$,
$\tau^{\Phi_2}_{11}$ --- наличие адекватной модели устаревания,
$\tau^{\Phi_2}_{12}$ --- количесто поддерживаемых параметров системы для оценки ТС,
$\tau^{\Phi_2}_{13}$ --- точность измерения и оценивания параметров для последующего расчета эффективности.

Ограничения: $\tau^{\Phi_2}_{12} \in [0,N]$,
$\tau^{\Phi_2}_{13} \in [0.001,1]$.

\begin{equation}
    \label{eq:formal_requirements_3}
    T^{\Phi_3}_1=[\tau^{\Phi_3}_{11},\tau^{\Phi_3}_{12}]
\end{equation}
где $T^{\Phi_3}_1$ --- требования класса $\Phi_3$,
$\tau^{\Phi_3}_{11}$ --- наличие адекватной модели деградации,
$\tau^{\Phi_3}_{12}$ --- поддержка возможности принятия решения о модернизации.

Ограничения: $\tau^{\Phi_3}_{11} \in [0,1]$,
$\tau^{\Phi_3}_{12} \in [0,1]$.

\begin{equation}
    \label{eq:formal_requirements_4}
    T^{\Phi_4}_1=[\tau^{\Phi_4}_{11}]
\end{equation}
где $T^{\Phi_4}_1$ --- требования класса $\Phi_3$,
$\tau^{\Phi_4}_{11}$ --- возможность анализа нескольких систем сразу.

Ограничения: $\tau^{\Phi_4}_{11} \in [0,1]$.

Формализуем систему в виде функции формирования нового информационного объекта.

Используем следующее обобщенное описание системы \ref{eq:formal_obobsh}: 

\begin{equation}
    \label{eq:formal_obobsh}
    S=(X,Y,Z,H,G, \left\{ t \right\}, \left\{ n \right\}, \left\{ r \right\} )
\end{equation}
где $X$ --- входы,
$Y$ --- выходы,
$Z$ --- состояния,
$H$ --- оператор переходов,
$G$ --- оператор выходов,
$t$ --- элементы,
$n$ --- свойства,
$r$ --- отношения между элементами.

Опишем функционирование информационной системы с заданными требованиями. 
Проектируемая система производит преобразование \ref{eq:formal_preobr}:
\begin{equation}
    \label{eq:formal_preobr}
    Y_t=F(X_t,T_t)
\end{equation} 
где $Y_t$ --- текущее состояние выходного объекта системы в момент времени $t$,
$F$ --- функция преобразования входного объекта системы,
$X_t$ --- текущее состояние входного объекта системы в момент времени $t$,
$T_t$ --- текущее требование к системе.

Можно определить закон функционирования рассматриваемой системы \ref{eq:formal_zakon}, который имеет имеет вид: 
\begin{equation}
    \label{eq:formal_preobr}
    \bigcup_{n=1}^N \left\{ X_i^{(n)} | (\delta_i \leq \delta^{X_i}) \right\}
    \xrightarrow{a_i \in A, l=\overline{1,n}}
    \bigcup_{n=1}^N \left\{ Y_i^{(n)} | (\theta_j \leq \theta^{Y_j}) \right\}
\end{equation} 
где $\bigcup_{n=1}^N \left\{ X_i^{(n)} \right\}$ --- объединение множеств входных объектов системы,
$\delta_i$ --- качество входных данных;
$\delta^{X_i}$ --- допустимое качество входных данных; 
$a_l$ --- внутренние состояния исходной системы; 
$\theta_j$ --- качество выходных данных;
$\theta^{Y_j}$ --- допустимое качество выходных данных. 

Поступающие на вход данные об анализируемой системе, в системе проходят предварительную обработку. 
Далее формируется модель устаревания системы, на основе которой рассчитываются параметры устаревания, прогнозируется значение эффективности системы и формируется решение о начале процесса реинжиниринга. 
На выход поступают результаты анализа и исследований. 
Данные результаты, совместно с предложенным системой решением, используются ЛПР для принятия окончательного решения.

\subsection*{Выводы раздела 1}

Можно сделать вывод, что в данном разделе был проведен анализ предметной области. 
Были определены основные математические модели процесса деградации системы. 
Построена модель процессов в СППР с использованием методологии IDEF0. 
Сформировано формализованное описание системы в виде совокупности элементов, ограничений и требований. 
В результате чего был определен закон функционирования системы.

% Конечной целью выпускной квалификационной работы магистра на тему <<Исследование процессов обеспечения безопасности облачных сред>> является повышение эффективности процессов обеспечения информационной безопасности облачных сред.
% Для достижения поставленной цели необходимо разработать структуру системы обеспечения информационной безопасности заданной облачной среды, провести внедрение и экспериментальные исследования.

% Для достижения целей исследования должны быть решены следующие задачи:
% \begin{itemize}
%   \item обзор составных частей облачной инфраструктуры;
%   \item анализ технологий используемых облачными провайдерами, необходимых для построения облачной инфраструктуры;
%   \item специфика применений облачных вычислений в России;
%   \item исследование проблемы безопасности облачных вычислений;
%   \item исследование уязвимостей в облачной среде, с использованием программ, распространяющихся под свободными лицензиями, например \hyperlink{gnu}{GNU} \hyperlink{gpl}{GPL}.
% \end{itemize}

% Входными данными для исследования являются облачная инфраструктура состоящая из физических и виртуальных серверов на базе операционной системы Linux.

% Выходными данными являются структура системы информационной безопасности и программное обеспечение для получения данных о уязвимостях в программном обеспечении, используемом в инфраструктуре, а также собранная и проанализированная информация о проблемах безопасности в облачной среде.

% Для применения практических навыков исследования уязвимостей необходима аппаратная платформа со следующими характеристиками:
% \begin{itemize}
%   \item процессор с поддержкой аппаратной виртуализации;
%   \item минимальный объем \hyperlink{ram}{ОЗУ} 8 Гб, рекомендуемый --- не менее 16 Гб;
%   \item минимум 15 Гб места на жестком диске (\hyperlink{ssd}{SSD});
%   \item операционная система Ubuntu 16.04, CentOS 7 или Debian 8 GNU/Linux.
% \end{itemize}

% Данная задача также рассматривается с точки зрения системного и вариантного анализа.

% Системный анализ включает в себя \cite{sys-analyz}:
% \begin{itemize}
%   \item системотехническое представление системы безопасности в виде <<черного ящика>>;
%   \item описание входных и выходных данных;
%   \item список функций, которые выполняет система безопасности;
%   \item учет случайностей;
%   \item декомпозицию системы и описание связей между ее элементами.
% \end{itemize}

% Вариантный анализ произведен исходя из выбранных критериев \cite{var-analyz}:
% \begin{itemize}
%   \item цена;
%   \item масштабируемость;
%   \item отказоустойчивость;
%   \item интерфейсы управления.
% \end{itemize}

\clearpage
