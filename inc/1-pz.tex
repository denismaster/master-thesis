\section{АНАЛИЗ ИНФОРМАЦИОННЫХ ПРОЦЕССОВ В ЗАДАЧЕ АНАЛИЗА ПРОЦЕССА ДЕГРАДАЦИИ}

В существующих системах в области анализа и совершенствования ТС напрямую не учитывается изменение эффективности во времени в процессе жизненного цикла. 
Используемые данными системами модели основываются на утверждении об экспоненциальном законе устаревания либо закон вообще не упоминают. 

Рассмотрим подробнее предметную область анализа процесса деградации и процессы, связанные с работой СППР в рамках данной области.

\subsection{Стандарты описания процессов в рамках прикладной предметной области}
Для того чтобы построить адекватную модель деградации реальной системы следует учитывать скорость изменения процесса изменения ИПЭ системы вследствие процесса деградации, а также готовность системы к реинжинирингу. 

В общем случае изменение ИПЭ системы во времени $E(t)$ представимо в виде:

\begin{equation} \label{eq:oldModel}
    E(t) = \begin{cases}
        E_0,            & \; t<t_m         \\
        E_0 \cdot f(t), & \; t<t_m         \\
        E_0 \cdot r,          & \; t>t_r, r>1 \\
    \end{cases}
\end{equation}

\begin{conditions*}
         E(t)       &   эффективность ТС в заданный момент времени \\
         E_0       &   начальная эффективность ТС \\
         t_m       &   время начала морального и\или технического устаревания системы \\
         t_r       &   время начала интенсивного совершенствования (модернизации) системы \\
         r         &   коэффициент реинжиниринга, который может быть ограничен возможным уровнем автоматизации системы, 
         f(t)       &  функция деградации.
\end{conditions*}

% где	E(t) — эффективность ТС в заданный момент времени, 
% E0 — начальная эффективность ТС,
% tm — время начала морального и\или технического устаревания системы,
% tr — время начала интенсивного совершенствования (реинжиниринга) системы, 
% r — коэффициент реинжиниринга, который может быть ограничен возможным уровнем автоматизации системы, 
% f(t) — функция устаревания.

На рисунке \ref{degradation_model} показана модель деградации комплексной системы. 
На первом отрезке $E(t) = const$. 
Это означает, что ИПЭ практически не изменяется. 
На отрезке t > tm показатель эффективности падает и определяется в общем случае некоторой функцией f(t). 
Третий отрезок характеризует период реинжиниринга, который повышает показатель эффективности по отношению к исходной эффективности системы (E0) [5].

\addimg{degradation_model}{0.6}{Модель изменения ИПЭ}{degradation_model}

Теоретический диапазон гибкости реинжиниринга иллюстрируется как площадь треугольника $ABC$. Верхняя часть треугольника может деформироваться и ограничиваться, например, экспонентой и определять фактический диапазон гибкости. При этом, изгиб экспоненты может быть различным, что оказывает существенное влияние на показатель эффективности ИС. Время принятия решения о начале реинжиниринга связано с выходом эффективности ИС на допустимую границу (заданную в документации) и определяется условием (1.2):

где	$E_{min}$ — минимальный допустимый уровень показателя эффективности системы.

В ходе анализа процесса устаревания возможно составить табличную функцию, описывающую динамику изменения эффективности ИС в процессе устаревания. 
Для упрощения процедуры анализа предлагается использовать аппроксимацию данной табличной функции.

Рассмотрим следующую классификацию технических систем в зависимости от функции деградации:

А - считать систему слабо устаревающей, если после начала морального устаревания (в период, равный 1 рабочему циклу системы) изменение
показателя ее эффективности не превышает 5-10 \%. 
На рисунке \ref{degradation_model} эта ситуация отражена функцией $E_0 \cdot e^{k(t−tm)}$;

Б - считать систему интенсивно устаревающей, если после начала устаревания (в период равный 1 рабочему циклу системы) изменение показателя ее эффективности превышает 20%;

В - считать систему замедленно устаревающей (системой высокой готовности). 

Классификация также отражена на рисунке \ref{classificationl}.

\addimg{classification}{1}{Условная классификация типов устаревания систем}{classification}
 
Разрабатываемся СППР должна давать возможность построения различные типы моделей деградации систем. 
Построение и применение адекватной модели деградации позволяет решать следующие задачи:
\begin{itemize}
  \item определение момента экономически целесообразного начала реинжиниринга;
  \item расчет эффективности ИС в заданный момент времени;
  \item расчет показателей устаревания ИС и подстройка параметров ИС под заданные показатели устаревания.
\end{itemize}

Таким образом, целесообразно определить модели устаревания ИС, основанные на различных законах распределения случайных величин, а также получить расчеты параметров эффективности системы при ее устаревании, на основе анализа зависимости этих параметров от вида закона распределения.

Проведем декомпозицию системы для структуризации целей, выявления проблем и противоречий. 
Построим функциональную модель СППР, отражающую структуру и функции системы [11,12,13]. 

Для исследования и проектирования систем на логическом уровне используется стандарт IDEF0. 
Данный стандарт часто используется для описания и оптимизации процессов в анализируемой системе. 

Построение модели IDEF0 начинается с представления всей системы в виде контекстной диаграммы.
Для построения модели использовался продукт Ramus Process Builder 1.2.5. 

Контекстная диаграмма верхнего уровня проектируемой системы представлена на рисунке \ref{idef0_context}.

\addimg{idef0_context}{1}{Контекстная диаграмма верхнего уровня}{idef0_context}

Расшифровка данной диаграммы представлена в таблице А.1.

Описание анализируемой системы обычно содержит в себе параметры системы, критерии, методику расчета эффективности. 
Также существует возможность отправки значений параметров системы автоматически, либо задания табличных значений ИПЭ во времени. 
В случае построения модели на основе табличных данных используется аппроксимация данного набора значений при помощи различных математических методов.


В качестве управляющих параметров может использоваться уже существующая модель устаревания ИС, необходимая для расчетов других параметров и принятия решения. 
Также она может быть уточнена в процессе работы системы. 
Также в вектор управляющих параметров входят и ограничения системы, если расчет ведется с целью определения параметров системы с заданной функцией устаревания.

В качестве механизма обычно используются различные вспомогательные программные средства для моделирования, например AnyLogic или Simulink, а также дополнительное ПО.

На выход процесса поступают обновленная модель устаревания, рассчитанные параметры процесса устаревания, решение о целесообразности проведение реконструкции(реинжиниринга) системы, а также параметры системы в случае их расчетов на основе заданной модели устаревания.

С помощью диаграммы декомпозиции первого уровня показана детализация процессов, протекающих при преобразовании главной функции системы. Это проиллюстрировано на рисунке \ref{idef0_decompose}.

В таблице А.2 приводится описание процессов, изображенных на данной диаграмме.

\addimg{idef0_decompose}{1}{Детализация процессов, протекающих при преобразовании главной функции системы}{idef0_decompose}

Модель системы, а также ее параметры и критерии поступают на вход процесса определения значения параметров системы. 
Результаты данных вычислений в виде вектора поступают на вход процесса расчета эффективности. 
Данный процесс использует модель деградации системы в качестве управления. 
В данном процессе происходит расчет интегрального показателя эффективности. 

В процессе уточнения модели системы определяется наиболее эффективный набор параметров, чтобы динамика изменения ИПЭ удовлетворяла заданным ограничениям. 
В качестве входных данных используются данные об изменении ИПЭ в течении заданного промежутка времени. 
Ограничения, накладываемые на задачу моделирования, используются в качестве управления. 
В качестве механизмов(ресурсов) используется вспомогательное ПО. 
Наиболее оптимальная модель деградации системы является выходным параметром.

Также полученная модель устаревания поступает в качестве входных данных в процесс поддержки принятия решений. 
В рамках данного процесса ЛПР предоставляются возможные варианты действий с системой.
Принятые решения являются выходными данными данного процесса.

% Конечной целью выпускной квалификационной работы магистра на тему <<Исследование процессов обеспечения безопасности облачных сред>> является повышение эффективности процессов обеспечения информационной безопасности облачных сред.
% Для достижения поставленной цели необходимо разработать структуру системы обеспечения информационной безопасности заданной облачной среды, провести внедрение и экспериментальные исследования.

% Для достижения целей исследования должны быть решены следующие задачи:
% \begin{itemize}
%   \item обзор составных частей облачной инфраструктуры;
%   \item анализ технологий используемых облачными провайдерами, необходимых для построения облачной инфраструктуры;
%   \item специфика применений облачных вычислений в России;
%   \item исследование проблемы безопасности облачных вычислений;
%   \item исследование уязвимостей в облачной среде, с использованием программ, распространяющихся под свободными лицензиями, например \hyperlink{gnu}{GNU} \hyperlink{gpl}{GPL}.
% \end{itemize}

% Входными данными для исследования являются облачная инфраструктура состоящая из физических и виртуальных серверов на базе операционной системы Linux.

% Выходными данными являются структура системы информационной безопасности и программное обеспечение для получения данных о уязвимостях в программном обеспечении, используемом в инфраструктуре, а также собранная и проанализированная информация о проблемах безопасности в облачной среде.

% Для применения практических навыков исследования уязвимостей необходима аппаратная платформа со следующими характеристиками:
% \begin{itemize}
%   \item процессор с поддержкой аппаратной виртуализации;
%   \item минимальный объем \hyperlink{ram}{ОЗУ} 8 Гб, рекомендуемый --- не менее 16 Гб;
%   \item минимум 15 Гб места на жестком диске (\hyperlink{ssd}{SSD});
%   \item операционная система Ubuntu 16.04, CentOS 7 или Debian 8 GNU/Linux.
% \end{itemize}

% Данная задача также рассматривается с точки зрения системного и вариантного анализа.

% Системный анализ включает в себя \cite{sys-analyz}:
% \begin{itemize}
%   \item системотехническое представление системы безопасности в виде <<черного ящика>>;
%   \item описание входных и выходных данных;
%   \item список функций, которые выполняет система безопасности;
%   \item учет случайностей;
%   \item декомпозицию системы и описание связей между ее элементами.
% \end{itemize}

% Вариантный анализ произведен исходя из выбранных критериев \cite{var-analyz}:
% \begin{itemize}
%   \item цена;
%   \item масштабируемость;
%   \item отказоустойчивость;
%   \item интерфейсы управления.
% \end{itemize}

\clearpage
