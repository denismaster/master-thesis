\section{Разработка системы поддержки принятия решений для исследования деградационных процессов СТО}

При реализации сервиса была использована платформа Node.js 11.0  и язык программирования Typescript 3.3. 
Использовалась среда разработки Visual Studio Code.

Для удобства кодирования сервис был разделен на несколько частей:

\begin{itemize}
    \item Клиентская часть;
    \item Серверная часть;
    \item База данных.
\end{itemize}
Для первоначальной настройки проекта использовался инструмент nest.js \cite{Nest}. 

Данный инструмент предоставляет большое количество инструментария для создания продвинутых веб-приложений с использованием самых современных техник разработки.

Интерфейс пользователя разработан с применением фреймворка React 16 \cite{React} и языка Typescript.

Для стилизации используется CSS-библиотека Ant Design \cite{Antd}.
В качестве базы данных используется СУБД MongoDB \cite{Mongo}.

Система разработана с поддержкой многопользовательской работы. 
Для этого в системе используется понятие \emph{проекта}. 
Каждый проект используется для хранения информации о конкретном исследовании и содержит одну или несколько моделей деградации. 
Добавление пользователей и проектов позволяет увеличить взаимодействие пользователей и эффективность анализа. 

После входа в систему пользователь попадает на страницу проектов.
На данной странице отображается список с каждым из проектов, разделенный на страницы по 20 элементов.
Для каждого проекта отображается его название, описание, а также набор тегов. 
Проекты могут быть приватными или публичными.

Также на данной странце присутствует функционал поиска, который позволяет фильтровать доступные проекты по названию или тегам.
Пользователь на данной странице может видеть все доступные ему проекты.

Пример страницы проектов показан на рисунке \ref{projects_page}.

\addimg{projects_page}{1}{Пример страницы проектов}{projects_page}

Также пользователь может создавать новые проекты. 
Для этого он может нажать кнопку "Создать проект". 
При этом откроется диалоговое окно создания нового проекта.

При создании нового проекта пользователь обязан указать его название. Описание при этом является опциональным полем.
Страница создания нового проекта представлена на рисунке \ref{new_project}.

\addimg{new_project}{1}{Пример диалогового окна создания проекта}{new_project}

Кликнув по строке с нужным проектом на странице проектов, пользователь попадает на страницу выбранного проекта.

Каждая страница проекта поделена на 3 области.

Область информации содержит основные свойства проекта, такие как метки и название.
Область критериев содержит панель управления критериями. 
Область графиков предназначена для визуализации построенной модели деградации системы.

Рассмотрим каждую из областей подробнее.

Область информации о проекте содержит название проекта, описание, а также пользовательские метки. 
Каждый из данных элементов является редактируемым, а все изменения сохраняются автоматически на сервере.
Также данная область содержит ссылку перехода на параметры проекта, а также кнопку "Сохранить", которая используется для сохранения состояния модели на сервере.
Данная область проиллюстрирована на рисунке \ref{project_panel_info}

\addimg{project_panel_info}{1}{Пример области информации о проекте}{project_panel_info}
 
Рассмотрим подробнее область критериев. Данная область использует вкладки для структуризации параметров.

На вкладке "Критерии" расположен список скалярных критериев в виде раскрывающегося списка. 
В каждой из панелей списка можно увидеть название, вес и дополнительные параметры критерия.

Доступны следующие законы изменения значений критериев в зависимости от времени:
\begin{itemize}
    \item константый --- значение критерия изменяется согласно константному закону;
    \item линейный --- значение критерия изменяется согласно линейному закону;
    \item экспоненциальный --- значение критерия изменяется согласно экспоненциальному закону;
    \item квадратичный --- значение критерия изменяется согласно квадратичному закону;
    \item кусочно-заданная функция --- служит для комбинации вышеуказанных видов зависимостей;
    \item набор данных --- значение критерия уже заранее определено в виде набора данных. Данный набор интерполируется сплайном;
    \item сплайн --- значение критерия изменяется согласно заранее заданному сплайну.
\end{itemize}

Каждый критерий связан с набором значений, рассчитанных в каждый момент времени $t$. 
Данные значения рассчитываются автоматически на основании выбранного закона, либо с помощью заданного набора данных.
При этом существует возможность автоматической отправки данных в критерий, реализованный по технологии \emph{webhook}.

При создании критерия предлагается ввести его название, его вес, выбрать закон изменения его значения в зависимости от времени, 
а также ввести дополнительные параметры, необходимые для конкретизации закона.

На рисунке \ref{project_new_criteria1} показано диалоговое окно создания критерия с константным законом изменения значения параметра.
На рисунке \ref{project_new_criteria3} показано диалоговое окно создания критерия с экспоненциальным законом изменения значения параметра.
На рисунке \ref{project_new_criteria4} показано диалоговое окно создания критерия с законом изменения значения на основе набора данных.

\addimg{project_new_criteria1}{0.65}{Пример диалогового окна создания критерия}{project_new_criteria1}
\addimg{project_new_criteria3}{0.65}{Пример диалогового окна создания критерия}{project_new_criteria3}
\addimg{project_new_criteria4}{0.65}{Пример диалогового окна создания критерия}{project_new_criteria4}

Так как критерии могут иметь различные единицы измерения своих значений, доступна нормализация критериев путем приведения к вероятностным характеристикам.

На основе заданного пользователем множества скалярных критериев можно сформировать соответствующий векторный критерий $K(t)=(K_1(T), K_2(t), \cdots, K_m(t))$, 
который будет характеризовать систему в каждый момент времени. 
На основании значений векторного критерия возможно произвести расчет интегрального показателя эффективности.

Параметры расчета ИПЭ представлены на вкладке "Параметры".
На этой вкладке можно задать основные параметры модели, требуемые для корректного расчета. 
В первой группе расположены селекторы дат начала и окончания процесса исследования, а также меню выбора шага.
При этом возможно не задавать дату окончания исследования.
Вышеуказанные параметры используются для определения модельного времени и корректных расчетов. 
Данная группа показана на рисунке \ref{project_time}.

\addimg{project_time}{1}{Пример задания параметров модельного времени}{project_time}

Следующй основной параметр, доступный для изменения --- минимальное допустимое значение ИПЭ. 
Данный параметр требуется для расчета оптимального момента модернизации системы.
Поле ввода данного параметра показано на рисунке \ref{project_mae}.

\addimg{project_mae}{1}{Пример задания минимального допустимого ИПЭ}{project_mae}

Последний основной параметр --- метод свертки скалярных критериев системы, необходимый для корректного расчета ИПЭ в каждый момент времени.
Доступны следующие свертки критериев:
\begin{itemize}
    \item аддитивная;
    \item мультипликативная;
    \item Чебышева.
\end{itemize}

Пример выбора вида свертки показан на рисунке \ref{project_conv}.

\addimg{project_conv}{1}{Пример выбора вида свертки критериев}{project_conv}

Слева расположена область графиков. На ней отображаются графики изменения значений критериев,
а также график изменения ИПЭ. Пример данной области представлен на рисунке \ref{project_chart}.

\addimg{project_chart}{1}{Пример графика изменения ИПЭ}{project_chart}

Для упрощения принятия решения в системе происходит расчет следующих параметров:
\begin{itemize}
    \item время начала процесса деградации;
    \item оптимальный момент начала модернизации.
\end{itemize}

Для поиска времени начала процесса деградации определяется экстремум функции деградации $f(t)$, 
а также точка перегиба путем численного расчета производных $f'(t)$ и $f''(t)$.
Пример отображения времени начала деградации показан на рисунке \ref{project_stat_begin}.

\addimg{project_stat_begin}{0.55}{Пример отображения времени начала деградации}{project_stat_begin}

Для поиска наиболее оптимального времени начала модернизации решается уравнение $M(t)=f(t)$. 
Пример отображения оптимального времени начала модернизации показан на рисунке \ref{project_stat_opt}.

\addimg{project_stat_opt}{0.55}{Пример отображения оптимального времени модернизации}{project_stat_opt}
 
Для решения задачи поиска наиболее оптимальной комбинации параметров, при которых динамика ИПЭ удовлетворяет заданным ограничениям, был разработан специальный интерфейс. 
В данном интерфейсе возможно задать несколько вариантов критериев, а также дополнительные ограничения.

Для поиска наиболее оптимального решения используется поиск методом перебора параметров. 
Составляются сочетания различных параметров друг с другом из их допустимых множеств. 
Далее производится расчет данных ИПЭ. 
Далее используется метод МНК для определения, удовлетворяет ли полученный набор данным заданным ограничениям.
Пример графика эффективности системы с областью ограничения представлен на рисунке \ref{project_modeling}.

\addimg{project_modeling}{1}{Пример интерфейса для поиска оптимальных сочетаний критериев}{project_modeling}

Исходный код разработанной системы представлен в приложении \hyperlink{app-a}{А}.

\anonsubsection{Выводы раздела 3}
В данном разделе был описан процесс разработки системы поддеркжи принятия решений для анализа процесса устаревания СТО.
Описаны основные возможности системы. 
Предложены алгоритмы работы ЛПР для анализа процессов деградации при исследовании СТО, а также при поиске оптимального процесса устаревания.
% В разделе описания облачной инфраструктуры представлена структурная схема облачной инфраструктуры, а также архитектура системы безопасности.

% \subsection{Структурная схема облачной инфраструктуры}

% Структурная схема облачной инфраструктуры представлена на рис. \ref{infrast-scheme}.

% \addimghere{infrast-scheme}{1}{Структурная схема облачной инфраструктуры}{infrast-scheme}

% В ЦОД 1 располагается основная часть инфраструктуры: сервера виртуального хостинга, виртуализации OpenVZ и KVM, сервер резервного копирования и выделенные сервера клиентов.

% В ЦОД 2 на двух арендованных виртуальных машинах находится один из подчиненных DNS-серверов, а также сервер мониторинга.
% На физических серверах располагаются важные элементы инфраструктуры: DNS-сервера, система биллинга и система управления IP-адресами.

% \subsection{Архитектура системы безопасности}

% Архитектура системы информационной безопасности представлена на рис. \ref{cwpp}.

% \addimghere{cwpp}{1}{Архитектура системы безопасности}{cwpp}

% Среди компонент и процессов систем защиты можно выделить несколько наиболее важных.
% Наиболее важными компонентами схемы являются:
% \begin{itemize}
%   \item контроль физического доступа;
%   \item регулярное обновление программного обеспечения;
%   \item централизованный мониторинг;
%   \item проведение тестов на поиск уязвимостей.
% \end{itemize}

\clearpage