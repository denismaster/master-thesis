\section{Описание системы информационной безопасности облачной среды}

\subsection{Применение сплайнов для аппроксимации процесса деградации ТС}
Визуализация процесса функционирования сложных систем, например, в рамках их деградации или устаревания, сопряжена с большим числом точек интерполяции, что порождает трудности при вычислениях. 
При разбиении отрезков интерполяции на несколько частей в точках сшивки разных интерполяционных полиномов будет разрывной их первая производная [2]. 

Для решения задачи кусочно-линейной интерполяции используют особый вид кусочно-полиномиальной интерполяции — сплайн-интерполяцию. 
Сплайн — это функция, которая на каждом частичном отрезке интерполяции является алгебраическим многочленом, а на всем заданном отрезке непрерывна вместе с несколькими своими производными [2, 3, 4, 5].

Для аппроксимации может быть использован метод кубических B-сплайнов(а), либо метод наименьших квадратов(б) [4,5].

Сплайны активно используются для аппроксимации временных рядов. 
Поэтому они могут быть использованы для анализа процесса деградации уже существующей системы [6]. 
Использование уже существующих наборов данных позволяет выполнить прогнозирование характера процесса деградации, либо определить наиболее оптимальный момент реинжиниринга.

Еще одной задачей, которая может быть решена сплайнами, является задача моделирования системы до запуска в эксплуатацию. 
Сплайны могут выступать в роли ограничений процесса устаревания сложной системы. 
Определив множество возможных диапазонов параметров, можно определить наиболее оптимальные параметры, при которых система ведет себя заданным образом.
Рассмотрим формальное определение сплайнов как математических объектов

\subsubsection{Определение сплайнов}

\subsubsection{Применение сплайнов для нахождения ИПЭ в заданный момент времени}
Достаточно часто возникает задача определить, а насколько эффективной будет система в любой момент времени. 

Для решения данной задачи необходимо заданный момент времени t подставить в уравнение сплайна, таким образом достаточно просто рассчитать следующее выражение:
\begin{equation}
    g(t), t \in [t_{min},t_{max}]
\end{equation}

Полученное численное значение будет искомым значением ИПЭ в заданный момент времени.

\subsubsection{Применение сплайнов для нахождения оптимального момента реинжиниринга}

Другой задачей является определение наиболее оптимальных моментов реинжиниринга.
Для этого вводят определение минимально-допустимой эффективности – уровня эффективности системы, являющегося граничным. 
Если значение эффективности системы меньше, чем этот показатель, то функционирование системы нежелательно или сопряжено с экономическими потерями. 
Наиболее оптимальный момент начала реинжиниринга – в момент, когда ИПЭ системы равен минимально-допустимой эффективности.

Рассмотрим формализованное решение задачи:
\begin{equation}
    \begin{cases}
        M(t)=M  \\
        g(t)=M(t), & t \in [t_{min},t_{max}]
    \end{cases}
\end{equation}

Решение данного уравнения и будет искомым моментом для начала реинжиниринга.


\subsubsection{Применение сплайнов для поиска наиболее оптимальных моделей с учетом ограничений}
% В разделе описания облачной инфраструктуры представлена структурная схема облачной инфраструктуры, а также архитектура системы безопасности.

% \subsection{Структурная схема облачной инфраструктуры}

% Структурная схема облачной инфраструктуры представлена на рис. \ref{infrast-scheme}.

% \addimghere{infrast-scheme}{1}{Структурная схема облачной инфраструктуры}{infrast-scheme}

% В ЦОД 1 располагается основная часть инфраструктуры: сервера виртуального хостинга, виртуализации OpenVZ и KVM, сервер резервного копирования и выделенные сервера клиентов.

% В ЦОД 2 на двух арендованных виртуальных машинах находится один из подчиненных DNS-серверов, а также сервер мониторинга.
% На физических серверах располагаются важные элементы инфраструктуры: DNS-сервера, система биллинга и система управления IP-адресами.

% \subsection{Архитектура системы безопасности}

% Архитектура системы информационной безопасности представлена на рис. \ref{cwpp}.

% \addimghere{cwpp}{1}{Архитектура системы безопасности}{cwpp}

% Среди компонент и процессов систем защиты можно выделить несколько наиболее важных.
% Наиболее важными компонентами схемы являются:
% \begin{itemize}
%   \item контроль физического доступа;
%   \item регулярное обновление программного обеспечения;
%   \item централизованный мониторинг;
%   \item проведение тестов на поиск уязвимостей.
% \end{itemize}

\clearpage