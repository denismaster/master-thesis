\section{Описание системы информационной безопасности облачной среды}

\subsection{Применение сплайнов для аппроксимации процесса деградации ТС}
Визуализация процесса функционирования сложных систем, например, в рамках их деградации или устаревания, сопряжена с большим числом точек интерполяции, что порождает трудности при вычислениях. 
При разбиении отрезков интерполяции на несколько частей в точках сшивки разных интерполяционных полиномов будет разрывной их первая производная [2]. 

Для решения задачи кусочно-линейной интерполяции используют особый вид кусочно-полиномиальной интерполяции — сплайн-интерполяцию. 
Сплайн — это функция, которая на каждом частичном отрезке интерполяции является алгебраическим многочленом, а на всем заданном отрезке непрерывна вместе с несколькими своими производными [2, 3, 4, 5].

Для аппроксимации может быть использован метод кубических B-сплайнов(а), либо метод наименьших квадратов(б) [4,5].

Сплайны активно используются для аппроксимации временных рядов. 
Поэтому они могут быть использованы для анализа процесса деградации уже существующей системы [6]. 
Использование уже существующих наборов данных позволяет выполнить прогнозирование характера процесса деградации, либо определить наиболее оптимальный момент реинжиниринга.

Еще одной задачей, которая может быть решена сплайнами, является задача моделирования системы до запуска в эксплуатацию. 
Сплайны могут выступать в роли ограничений процесса устаревания сложной системы. 
Определив множество возможных диапазонов параметров, можно определить наиболее оптимальные параметры, при которых система ведет себя заданным образом.
Рассмотрим формальное определение сплайнов как математических объектов

\subsubsection{Определение сплайнов}

\subsubsection{Применение сплайнов для нахождения ИПЭ в заданный момент времени}
Достаточно часто возникает задача определить, а насколько эффективной будет система в любой момент времени. 

Для решения данной задачи необходимо заданный момент времени t подставить в уравнение сплайна, таким образом достаточно просто рассчитать следующее выражение:
\begin{equation}
    g(t), t \in [t_{min},t_{max}]
\end{equation}

Полученное численное значение будет искомым значением ИПЭ в заданный момент времени.

\subsubsection{Применение сплайнов для нахождения оптимального момента реинжиниринга}

Другой задачей является определение наиболее оптимальных моментов реинжиниринга.
Для этого вводят определение минимально-допустимой эффективности – уровня эффективности системы, являющегося граничным. 
Если значение эффективности системы меньше, чем этот показатель, то функционирование системы нежелательно или сопряжено с экономическими потерями. 
Наиболее оптимальный момент начала реинжиниринга – в момент, когда ИПЭ системы равен минимально-допустимой эффективности.

Рассмотрим формализованное решение задачи:
\begin{equation}
    \begin{cases}
        M(t)=M  \\
        g(t)=M(t), & t \in [t_{min},t_{max}]
    \end{cases}
\end{equation}

Решение данного уравнения и будет искомым моментом для начала реинжиниринга.


\subsubsection{Применение сплайнов для поиска наиболее оптимальных моделей с учетом ограничений}

\subsection{Описание системы поддержки принятия решения}
При реализации сервиса была использована платформа Node.js 11.0 и язык программирования Typescript 3.3. 
Использовалась среда разработки Visual Studio Code.

Для удобства кодирования сервис был разделен на несколько частей:
	Клиентская часть;
	Серверная часть;
	База данных.

Для первоначальной настройки проекта использовался инструмент nest.js. 
Данный инструмент предоставляет большое количество инструментария для создания продвинутых веб-приложений с использованием самых современных техник разработки.

Интерфейс пользователя разработан с применением фреймворка React 16 и языка Typescript.

Для стилизации используется CSS-библиотека Ant Design.

После входа пользователь может зайти на страницу проектов. 
Каждый проект используется для хранения информации о конкретном проекте и содержит одну или несколько моделей устаревания. 
Добавление пользователей и проектов позволяет увеличить взаимодействие пользователей и эффективность анализа. 
Пример страницы проектов показан на рисунке 3.1.

Также пользователь может создавать новые проекты. 
Страница создания нового проекта представлена на рисунке 3.2.


Рисунок 3.1 – Пример страницы проектов
 
Рисунок 3.2 – Пример страницы создания проекта

Кликнув по строке с нужным проектом, пользователь попадает на страницу проекта. 
Область информации о проекте содержит название проекта, описание, а также пользовательские метки. 
При этом существует возможность динамически изменять соответствующие параметры проекта. 
Пример области информации представлен на рисунке 3.3.

 
Рисунок 3.3 – Навигационное меню

Рабочая область разделена на две части. 
Слева располагается график изменения ИПЭ, а справа располагается область параметров проекта. 
Пример рабочей области показан на рисунке 3.4.

 
Рисунок 3.4 – Рабочая область

 
Рисунок 3.5 – Основные параметры проекта
Основные параметры проекта располагаются во вкладке «Параметры» и представлены на рисунке 3.5.

Пользователь имеет возможность определить минимальную и максимальную допустимые значения ИПЭ, определить метод свертки критериев, а также ограничить моделирование путем задания временного интервала. 
При этом пользователь может также выбрать шаг. 
Данные параметры напрямую влияют на модельное время и, соответственно, на расчеты.

Во вкладке «Критерии» располагаются параметры для каждого из критериев. 
Рассмотрим ее подробнее.

Для создания критерия разработан диалог «Новый критерий». 
Он показан на рисунке 3.6.

 
Рисунок 3.6 – Диалог создания критерия

Доступные типы критериев показаны на рисунке 3.7.
 
Рисунок 3.7 – Доступные виды критериев

При создании критерия необходимо задать его вес, который будет в дальнейшем использоваться при расчете ИПЭ путем применения математической свертки.
При этом данные веса будут валидироваться, т.к. их сумма должна быть всегда равна единице.

Рисунок 3.8 – Настройки критериев

На рисунке 3.8 показаны примеры настроек критериев. 
Можно отредактировать название критерия, его вес, а также дополнительные параметры.

Примеры графиков показателей представлены на рисунке 3.9-3.10.
На рисунке 3.11 показан интерфейс для нормализации критериев. 
Так как критерии системы могут быть неоднородными, то необходимо их привести к однородному виду. 
Для этого и предназначен процесс нормализации. 


Рисунок 3.9 – Пример графика изменения значения показателей по одному критерию

Рисунок 3.10 – Пример графика изменения интегрального показателя эффективности

Рисунок 3.11 – Пример настроек нормализации данных для критериев

Для решения задачи поиска наиболее оптимальной комбинации параметров, при которых динамика ИПЭ удовлетворяет заданным ограничениям, был разработан специальный интерфейс. 
В данном интерфейсе возможно задать несколько вариантов критериев, а также дополнительные ограничения.

Рисунок 3.12 – Пример интерфейса добавления нескольких вариантов критериев

Рисунок 3.13 – Пример интерфейса добавления ограничений

Для поиска наиболее оптимального решения используется поиск методом перебора параметров. 
Составляются сочетания различных параметров друг с другом из их допустимых множеств. 
Далее производится расчет данных ИПЭ. Далее используется метод МНК для определения, удовлетворяет ли полученный набор данным заданным ограничениям.
Пример графика эффективности системы с областью ограничения представлен на рисунке 3.14.

Рисунок 3.14– Интерфейс поиска наиболее оптимальных параметров с учетом ограничений

% В разделе описания облачной инфраструктуры представлена структурная схема облачной инфраструктуры, а также архитектура системы безопасности.

% \subsection{Структурная схема облачной инфраструктуры}

% Структурная схема облачной инфраструктуры представлена на рис. \ref{infrast-scheme}.

% \addimghere{infrast-scheme}{1}{Структурная схема облачной инфраструктуры}{infrast-scheme}

% В ЦОД 1 располагается основная часть инфраструктуры: сервера виртуального хостинга, виртуализации OpenVZ и KVM, сервер резервного копирования и выделенные сервера клиентов.

% В ЦОД 2 на двух арендованных виртуальных машинах находится один из подчиненных DNS-серверов, а также сервер мониторинга.
% На физических серверах располагаются важные элементы инфраструктуры: DNS-сервера, система биллинга и система управления IP-адресами.

% \subsection{Архитектура системы безопасности}

% Архитектура системы информационной безопасности представлена на рис. \ref{cwpp}.

% \addimghere{cwpp}{1}{Архитектура системы безопасности}{cwpp}

% Среди компонент и процессов систем защиты можно выделить несколько наиболее важных.
% Наиболее важными компонентами схемы являются:
% \begin{itemize}
%   \item контроль физического доступа;
%   \item регулярное обновление программного обеспечения;
%   \item централизованный мониторинг;
%   \item проведение тестов на поиск уязвимостей.
% \end{itemize}

\clearpage