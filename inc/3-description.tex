\section{Описание системы информационной безопасности облачной среды}

\subsection{Применение сплайнов для аппроксимации процесса деградации ТС}
Визуализация процесса функционирования сложных систем, например, в рамках их деградации или устаревания, сопряжена с большим числом точек интерполяции, что порождает трудности при вычислениях. 
При разбиении отрезков интерполяции на несколько частей в точках сшивки разных интерполяционных полиномов будет разрывной их первая производная [2]. 

Для решения задачи кусочно-линейной интерполяции используют особый вид кусочно-полиномиальной интерполяции — сплайн-интерполяцию. 
Сплайн — это функция, которая на каждом частичном отрезке интерполяции является алгебраическим многочленом, а на всем заданном отрезке непрерывна вместе с несколькими своими производными [2, 3, 4, 5].

Для аппроксимации может быть использован метод кубических B-сплайнов(а), либо метод наименьших квадратов(б). 
Пример данных сплайнов показан на рисунке \ref{splines}.

\addimg{splines}{0.7}{Различные методы аппроксимации при помощи сплайнов}{splines}

Сплайны активно используются для аппроксимации временных рядов.
Это показано на рисунке \ref{splines2}. 
Поэтому они могут быть использованы для анализа процесса деградации уже существующей системы [6]. 
Использование уже существующих наборов данных позволяет выполнить прогнозирование характера процесса деградации, либо определить наиболее оптимальный момент реинжиниринга.

\addimg{splines2}{0.7}{Применение сплайнов для аппроксимации временных рядов}{splines2}

Еще одной задачей, которая может быть решена сплайнами, является задача моделирования системы до запуска в эксплуатацию. 
Сплайны могут выступать в роли ограничений процесса устаревания сложной системы. 
Определив множество возможных диапазонов параметров, можно определить наиболее оптимальные параметры, при которых система ведет себя заданным образом.
Рассмотрим формальное определение сплайнов как математических объектов

\subsubsection{Определение сплайнов}
Математически говоря, для определения сплайна требуется несколько элементов: степень сплайна, набор смежных интервалов и, конечно, базовые многочлены на каждом из этих интервалов.
Степень сплайна, обозначенная в этом разделе как $ d $, является максимальной степенью базовых полиномов.
Также можно говорить о порядке сплайна, который составляет $ d + 1 $ для сплайна степени $ d $.
Интервалы определяются с помощью последовательности узлов $ k_0 <\ldots <k_n $, которая является строго возрастающей последовательностью вещественных чисел $ n + 1 $.
Значения $ k_0, \ldots, k_n $ называются \emph{узлами} (knots).
Эти узлы определяют $ n $ \emph{интервалов узлов} $ [k_i, k_ {i + 1}] $ для $ i \in [0, n-1] $.
На каждом интервале узлов сплайн степени $ d $ определяется полиномом степени не более $ d $.

\emph{Область естественного определения} сплайна - это интервал $ [k_0, k_n] $ \cite{Splines}.
Например, сплайн $s: [k_0, k_n] \rightarrow \mathbb {R} $ может быть определен как:
\begin{equation}
    s(x) =
    \sum_{i=0}^d p_{ij} (x - k_j)^i \qquad
    \begin{cases}
        x \in [k_j,k_{j+1}[ \textrm{ and } j \in [ 0,n-2 ] \\
        x \in [k_{n-1},k_n]
    \end{cases}
    \end{equation}

Сплайны, которые выражены в виде линейной комбинации полиномов третьей степени, называются кубическими сплайнами.
Сплайны также могут быть выражены в виде линейной комбинации B-сплайнов.
B-сплайны - это набор кусочно-полиномиальных функций, которые определяют подходящий базис для векторного пространства сплайнов \cite{Splines}.

B-сплайн $N_{i, d + 1} $ степени $ d $ (порядок $ d + 1 $) с узлами $ k_i < \ldots <k_ {i + d + 1} $ определяется рекурсивно при помощи выражений \ref{eq:splinesdef1} и \ref{eq:splinesdef2}:

\begin{equation} \label{eq:splinesdef1}
    N_{i,d+1}(x) = \frac{x - k_i}{k_{i+d} - k_i} N_{i,d}(x) + \frac{k_{i+d+1} - x}{k_{i+d+1} - k_{i+1}} N_{i+1,d}(x),
    \end{equation}
    \begin{equation} \label{eq:splinesdef2}
    N_{i,1}(x)= 
    \begin{cases} 
    1, & x \in [k_i,k_{i+1}] \\ 
    0, & x \notin [k_i,k_{i+1}]
    \end{cases}
    \end{equation}

Для интерполяции обычно используются кубические сплайны, тогда как B-сплайны могут быть использованы для построения ограничений функции деградации.

Рассмотрим процесс интерполяции на основе сплайнов.
На каждом отрезке $[x_{i - 1},x_{i}]$ можно определить функцию $S(x)$.
Данная функция будет являться полиномом третьей степени $S_i(x)$, коэффициенты которого должны быть определены в процессе интерполяции. 
Представим $S_i(x)$ в виде следующего выражения \ref{eq:cubspline1}:

\begin{equation} \label{eq:cubspline1}
    S_i(x) = a_i + b_i(x - x_i) + {c_i}(x-x_i)^2 + {d_i}(x - x_i)^3
\end{equation}

Тогда можно определить следующие выражения \ref{eq:cubspline2}
\begin{equation} \label{eq:cubspline1}
    S_i\left(x_i\right) = a_i, \quad S'_i(x_i) = b_i, \quad S''_i(x_i) = 2c_i
\end{equation}

Все производные до второго порядка включительно должны удовлетворять следующим выражениям:
\begin{equation} \label{eq:cubspline3}
    S_i\left(x_{i-1}\right) = S_{i-1}(x_{i-1})
\end{equation}
\begin{equation} \label{eq:cubspline4}
    S'_i\left(x_{i-1}\right) = S'_{i-1}(x_{i-1})
\end{equation}
\begin{equation} \label{eq:cubspline5}
    S''_i\left(x_{i-1}\right) = S''_{i-1}(x_{i-1})
\end{equation}

Условие интерполяции допустимо записать в следующем виде \ref{eq:cubspline6}:
\begin{equation} \label{eq:cubspline6}
    S_i\left(x_{i}\right) = f(x_{i})
\end{equation}

Введем новые обозначения: $\quad h_i = x_i - x_{i-1}, \quad f_{i} = f(x_{i})$. 
Отсюда получаем формулы для вычисления коэффициентов интерполяционного сплайна $S(x)$:
\begin{equation} \label{eq:cubspline7}
\begin{cases} 
a_{i} = f(x_{i}), \\ 
d_{i} = \frac{c_{i} - c_{i - 1}}{3 \cdot h_{i}} \\
b_{i} = \frac{a_{i} - a_{i - 1}}{h_{i}} + \frac{2 \cdot c_{i} + c_{i - 1}}{3} \cdot h_{i} \\
c_{i - 1} \cdot h_{i} + 2 \cdot c_{i} \cdot(h_{i} + h_{i+1}) + c_{i + 1} \cdot h_{i+1} = 3 \cdot \left(\frac{a_{i+1} - a_{i}}{h_{i+1}} - \frac{a_{i} - a_{i - 1}}{h_{i}}\right) \\
c_{N} = S''(x_{N}) = 0 \\
c_{1} - 3 \cdot d_{1} \cdot h_{1} = S''(x_{0}) = 0
0, & x \notin [k_i,k_{i+1}]
\end{cases}
\end{equation}

Пусть $c_{0} = c_{N} = 0$, тогда расчет интерполяцуонного сплайна можно произвести методом трехдиагональной матрицы.

Пусть $g(t)$ - сплайн, в качестве параметра которого используется координата времени. Рассмотрим его применение для решения поставленных задач.

\subsubsection{Применение сплайнов для нахождения ИПЭ в заданный момент времени}
Достаточно часто возникает задача определить, а насколько эффективной будет система в любой момент времени. 

Для решения данной задачи необходимо заданный момент времени t подставить в уравнение сплайна, таким образом достаточно просто рассчитать следующее выражение:
\begin{equation}
    g(t), t \in [t_{min},t_{max}]
\end{equation}

Полученное численное значение будет искомым значением ИПЭ в заданный момент времени.

\subsubsection{Применение сплайнов для нахождения оптимального момента реинжиниринга}

Другой задачей является определение наиболее оптимальных моментов реинжиниринга.
Для этого вводят определение минимально-допустимой эффективности – уровня эффективности системы, являющегося граничным. 
Если значение эффективности системы меньше, чем этот показатель, то функционирование системы нежелательно или сопряжено с экономическими потерями. 
Наиболее оптимальный момент начала реинжиниринга – в момент, когда ИПЭ системы равен минимально-допустимой эффективности.

Рассмотрим формализованное решение задачи:\begin{equation}
    \begin{cases}
        M(t)=M  \\
        g(t)=M(t), & t \in [t_{min},t_{max}]
    \end{cases}
\end{equation}

Решение данного уравнения и будет искомым моментом для начала реинжиниринга.


\subsubsection{Применение сплайнов для поиска наиболее оптимальных моделей с учетом ограничений}
В процессе поиска наиболее оптимальной модели ЛПР может варьировать ее структуру.

Так как не для каждой из системы возможно автоматически определить диапазоны допустимых параметров, 
в силу неоднородности структуры системы, то данные множества допустимых параметров формирует ЛПР.

Пусть формированы множества различных вариантов критериев в зависимости от сущности критерия:
\begin{equation}
    k^j=\left\{ k_1^j,k_2^j, \cdots ,k_n^j \right\}
\end{equation}

Формируем множество сочетаний $P$: 
\begin{equation}
    P=\left\{ (k^1_1, k^2_1, \cdots\, k^m_1), \cdots , (k^1_{K^1_{max}}, k^2_{K^2_{max}}, \cdots\, k^m_{K^m_{max}}) \right\}
\end{equation}
При этом мощность данного множества равна числу сочетаний: $ \norm{P} = K^1_{max} \cdot \ K^2_{max} \cdots  K^m_{max} $

Сформированное множество сочетаний формирует множество различных вариантов структуры системы.

Введем множество ограничений B
\begin{equation}
    B= \left\{ B_1(t),\cdots\, B_b(t) \right\} 
\end{equation}

При этом каждый из этих ограничений может быть сплайном.

Далее мы для каждого из сочетаний рассчитываем ИПЭ в заданные моменты времени и проверяем в каждый момент времени, удовлетворяет ли полученное ИПЭ требуемым ограничениям.

Для этого рассчитывается ошибка по методу наименьших квадратов. 
Если она в пределах e∈[0,e_max ], то данный набор параметров признается удовлетворяющим ограничениям.

Данный алгоритм позволяет определить допустимые сочетания параметров системы, при которых система удовлетворяет заданным ограничениям. 
Очевидно, что такое сочетание параметров, которое будет давать наименьшую ошибку, и будет искомым наиболее оптимальным.

\subsection{Описание системы поддержки принятия решения}
При реализации сервиса была использована платформа Node.js 11.0 и язык программирования Typescript 3.3. 
Использовалась среда разработки Visual Studio Code.

Для удобства кодирования сервис был разделен на несколько частей:

\begin{itemize}
    \item Клиентская часть;
    \item Серверная часть;
    \item База данных.
\end{itemize}
Д
ля первоначальной настройки проекта использовался инструмент nest.js. 

Данный инструмент предоставляет большое количество инструментария для создания продвинутых веб-приложений с использованием самых современных техник разработки.

Интерфейс пользователя разработан с применением фреймворка React 16 и языка Typescript.

Для стилизации используется CSS-библиотека Ant Design.

Система разработана с поддержкой многопользовательской работы. 
Для этого в системе используется понятие \emph{проекта}. 
Каждый проект используется для хранения информации о конкретном исследовании и содержит одну или несколько моделей деградации. 
Добавление пользователей и проектов позволяет увеличить взаимодействие пользователей и эффективность анализа. 

После входа в систему пользователь попадает на страницу проектов.
На данной странице отображается список с каждым из проектов, разделенный на страницы по 20 элементов.
Для каждого проекта отображается его название, описание, а также набор тегов. 
Проекты могут быть приватными или публичными.

Также на данной странце присутствует функционал поиска, который позволяет фильтровать доступные проекты по названию или тегам.
Пользователь на данной странице может видеть все доступные ему проекты.

Пример страницы проектов показан на рисунке \ref{projects_page}.

\addimg{projects_page}{0.8}{Пример страницы проектов}{projects_page}

Также пользователь может создавать новые проекты. 
Для этого он может нажать кнопку "Создать проект". 
При этом откроется диалоговое окно создания нового проекта.

При создании нового проекта пользователь обязан указать его название. Описание при этом является опциональным полем.
Страница создания нового проекта представлена на рисунке 3.2.

\addimg{new_project}{0.8}{Пример диалогового окна создания проекта}{new_project}

Кликнув по строке с нужным проектом на странице проектов, пользователь попадает на страницу выбранного проекта.

Каждая страница проекта поделена на 3 области.

Область информации содержит основные свойства проекта, такие как метки и название.
Область критериев содержит панель управления критериями. 
Область графиков предназначена для визуализации построенной модели деградации системы.

Рассмотрим каждую из областей подробнее.

Область информации о проекте содержит название проекта, описание, а также пользовательские метки. 
Каждый из данных элементов является редактируемым, а все изменения сохраняются автоматически на сервере.
Также данная область содержит ссылку перехода на параметры проекта, а также кнопку "Сохранить", которая используется для сохранения состояния модели на сервере.
Данная область проиллюстрирована на рисунке \ref{project_panel_info}

\addimg{project_panel_info}{0.9}{Пример области информации о проекте}{project_panel_info}
 
Рабочая область разделена на две части. 
Слева располагается график изменения ИПЭ, а справа располагается область параметров проекта. 
Пример рабочей области показан на рисунке 3.4.

 
Рисунок 3.4 – Рабочая область

 
Рисунок 3.5 – Основные параметры проекта
Основные параметры проекта располагаются во вкладке «Параметры» и представлены на рисунке 3.5.

Пользователь имеет возможность определить минимальную и максимальную допустимые значения ИПЭ, определить метод свертки критериев, а также ограничить моделирование путем задания временного интервала. 
При этом пользователь может также выбрать шаг. 
Данные параметры напрямую влияют на модельное время и, соответственно, на расчеты.

Во вкладке «Критерии» располагаются параметры для каждого из критериев. 
Рассмотрим ее подробнее.

Для создания критерия разработан диалог «Новый критерий». 
Он показан на рисунке 3.6.

 
Рисунок 3.6 – Диалог создания критерия

Доступные типы критериев показаны на рисунке 3.7.
 
Рисунок 3.7 – Доступные виды критериев

При создании критерия необходимо задать его вес, который будет в дальнейшем использоваться при расчете ИПЭ путем применения математической свертки.
При этом данные веса будут валидироваться, т.к. их сумма должна быть всегда равна единице.

Рисунок 3.8 – Настройки критериев

На рисунке 3.8 показаны примеры настроек критериев. 
Можно отредактировать название критерия, его вес, а также дополнительные параметры.

Примеры графиков показателей представлены на рисунке 3.9-3.10.
На рисунке 3.11 показан интерфейс для нормализации критериев. 
Так как критерии системы могут быть неоднородными, то необходимо их привести к однородному виду. 
Для этого и предназначен процесс нормализации. 


Рисунок 3.9 – Пример графика изменения значения показателей по одному критерию

Рисунок 3.10 – Пример графика изменения интегрального показателя эффективности

Рисунок 3.11 – Пример настроек нормализации данных для критериев

Для решения задачи поиска наиболее оптимальной комбинации параметров, при которых динамика ИПЭ удовлетворяет заданным ограничениям, был разработан специальный интерфейс. 
В данном интерфейсе возможно задать несколько вариантов критериев, а также дополнительные ограничения.

Рисунок 3.12 – Пример интерфейса добавления нескольких вариантов критериев

Рисунок 3.13 – Пример интерфейса добавления ограничений

Для поиска наиболее оптимального решения используется поиск методом перебора параметров. 
Составляются сочетания различных параметров друг с другом из их допустимых множеств. 
Далее производится расчет данных ИПЭ. Далее используется метод МНК для определения, удовлетворяет ли полученный набор данным заданным ограничениям.
Пример графика эффективности системы с областью ограничения представлен на рисунке 3.14.

Рисунок 3.14– Интерфейс поиска наиболее оптимальных параметров с учетом ограничений

% В разделе описания облачной инфраструктуры представлена структурная схема облачной инфраструктуры, а также архитектура системы безопасности.

% \subsection{Структурная схема облачной инфраструктуры}

% Структурная схема облачной инфраструктуры представлена на рис. \ref{infrast-scheme}.

% \addimghere{infrast-scheme}{1}{Структурная схема облачной инфраструктуры}{infrast-scheme}

% В ЦОД 1 располагается основная часть инфраструктуры: сервера виртуального хостинга, виртуализации OpenVZ и KVM, сервер резервного копирования и выделенные сервера клиентов.

% В ЦОД 2 на двух арендованных виртуальных машинах находится один из подчиненных DNS-серверов, а также сервер мониторинга.
% На физических серверах располагаются важные элементы инфраструктуры: DNS-сервера, система биллинга и система управления IP-адресами.

% \subsection{Архитектура системы безопасности}

% Архитектура системы информационной безопасности представлена на рис. \ref{cwpp}.

% \addimghere{cwpp}{1}{Архитектура системы безопасности}{cwpp}

% Среди компонент и процессов систем защиты можно выделить несколько наиболее важных.
% Наиболее важными компонентами схемы являются:
% \begin{itemize}
%   \item контроль физического доступа;
%   \item регулярное обновление программного обеспечения;
%   \item централизованный мониторинг;
%   \item проведение тестов на поиск уязвимостей.
% \end{itemize}

\clearpage