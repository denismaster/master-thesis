\section{ПРОЕКТИРОВАНИЕ И СИСТЕМОТЕХНИЧЕСКИЙ АНАЛИЗ ИНФОРМАЦИОННОЙ СИСТЕМЫ}

Рассмотрим разрабатываему СППР иерархически на различных уровнях абстракции. 
В зависимости от конкретных задач будет изменяться степень детализации анализируемого объекта. 
Для ряда случаев данная задача достаточно нетривиальна и требует использования специализированного подхода. 
Системный анализ является одним из эффективных и современных подходов к анализу сложных систем с точки зрения их проектирования.

Одним из наиболее активно используемых подходов анализа и синтеза комплексных технических систем является системотехнический подход. 
Одним из ключевых особенностей метода является определение структуры системы в виде множества элементов, а также множества связей между ними, которые отражают их взаимодействие. 
По данным множествам можно сделать вывод о структуре разрабатываемой системы.

Принципы системного анализа представляют собой некоторые обобщенные положения, которые интегрируют знания и опыт работы с комплексными системами. 
Следующие принципы считаются основополагающими в системном подходе: 

\begin{itemize}
    \item принцип конечной цели;
    \item принцип измерения;
    \item принцип единства;
    \item принцип связности;
    \item принцип модульности;
    \item принцип иерархии;
    \item принцип функциональности;
    \item надежность системы и расчетов;
    \item принцип развития;
    \item принцип сочетания централизации и децентрализации;
    \item принцип учета неопределенности и случайностей;
    \item принцип учета неопределенности и случайностей.
\end{itemize}

Пренебрежение данными принципами в ходе процесса проектирования любой комплексной технической или программной системы обязательно приведет к снижению ее эффективности.

Проведем более детальный анализ нашей системы. 
Для этого необходимо рассмотреть ее с точки зрения каждого принципа системного анализа из вышеуказанных. 
Данный подход позволит увеличить качество и эффективность разрабатываемой системы

Основные принципы системного анализа и их приложение к исследуемой системе рассмотрены ниже.

\subsection{Принцип конечной цели}
Принцип конечной цели – основополагающий принцип системного анализа.

Конечная цель имеет абсолютный приоритет, в системе все должно быть подчинено ее достижению. 

В соответствии с данным принципом должна быть четко сформулирована конечная цель – назначение проектируемой системы и сформирован список функций, которые должна выполнять система.

Представим разрабатываемую систему в виде «черного ящика». 
Структура представлена на рисунке \ref{black_box}.

\addimg{black_box}{0.6}{Структура разрабатываемой системы в виде черного ящика}{black_box}

На структуре вектор $X$ – входные данные, которые включают в себя следующую информацию:
\begin{itemize}
    \item описание анализируемой системы;
    \item описание параметров системы (начальная эффективность, минимально-допустимая эффективность, динамика изменения эффективности во времени и т.п.);
    \item параметры расчета (момент времени для расчета эффективности, запрос на решение о реинжиниринге) и т.д.
\end{itemize}

Вектор $Z$ – управляющие параметры, которые включают в себя следующую информацию:
\begin{itemize}
    \item нормативная база (ГОСТы и стандарты, нормативно-технические документы, которые устанавливают ряд норм, требований и правил) к анализируемой системе;
    \item техническое задание от заказчика, в котором прописаны все требования к конечному результату.
\end{itemize}
	
Вектор $W$ – механизм, который включают в себя следующую информацию:

\begin{itemize}
    \item средства моделирования.
\end{itemize}

$Y$ – выходные данные, которые включают в себя следующую информацию:
\begin{itemize}
    \item предлагаемое решение о начале реинжиниринга;
    \item значение эффективности системы в заданный момент времени;
    \item параметры системы, рассчитанные с учетом ограничений и заданной модели устаревания;
    \item модель устаревания анализируемой системы.
\end{itemize}
	
Для выполнения равенства $Y=F(X,Z,W)$  процесс должен выполнять следующие функции:
\begin{itemize}
    \item $Ф_1$– определение адекватной модели устаревания системы;
    \item $Ф_2$ – определение эффективности системы в заданный момент времени;
    \item $Ф_3$ – принятие решения о реинжиниринге;
    \item $Ф_4$ – поддержка анализа нескольких систем.
\end{itemize}

Функция $Ф_1$ является базовой функцией системы, она необходима для реализации иных функций.

Процесс, протекающий в проектируемой системе, можно разбить на несколько этапов. В промежутки времени между смежными этапами состояние системы не меняется. 

Основными этапами существования системы являются:
\begin{itemize}
    \item $\Phi_1$– выбор модели и ввод исходных данных в систему и ее запуск; 
    \item $\Phi_2$ – очередь;
    \item $\Phi_3$ – обработка заявки;
    \item $\Phi_4$ – выходной поток.
\end{itemize}
	 
Типичный сценарий работы системы: 
\begin{enumerate}
    \item На вход системы пользователь вводит все необходимые параметры;
    \item Настройка модели устаревания;
    \item Запуск расчетов;
    \item Расчет параметров системы, связанных с процессом деградации (экономически оптимальный срок эксплуатации системы до реинжиниринга, максимально отсроченное время начала реинжиниринга, темп морального устаревания системы и другие);
    \item Графическое представление результатов (модели изменения во времени эффективности информационной системы вследствие ее устаревания) в виде визуализации с использованием сплайнов;
    \item Определение целесообразности реконструкции системы, либо дальнейшего её использования. 
\end{enumerate}

\subsection{Принцип единства}
Принцип единства – это совместное рассмотрение системы как целого и как совокупности частей.

Принцип ориентирован на декомпозицию с сохранением целостных представлений о системе.

Система должна рассматриваться и как целое, и как совокупность элементов.

Расчленение системы необходимо производить, сохраняя целостное представление о системе.

Принцип подразумевает выделение подсистем, композиция которых в совокупности со связями позволяет выполнять все функции проектируемой системы, определяет ее структуру и может быть рассмотрена как единая, целостная система.

На основании функций проектируемой системы, представленных выше, в ней можно выделить следующие подсистемы:
\begin{itemize}
    \item подсистема ввода параметров модели;
    \item подсистема расчета модели устаревания ИС;
    \item подсистема анализа и принятия решений;
    \item подсистема графики и совместной работы.
\end{itemize}

На рисунке \ref{subsystem_scheme} представлена схема взаимодействия между подсистемами.

\addimg{subsystem_scheme}{0.7}{Схема взаимодействия между подсистемами}{subsystem_scheme}

Обозначения, приведенные на рисунке 2.2 требуют пояснения:
\begin{itemize}
    \item a – информация об анализируемой системе предоставляется пользователем в систему;
    \item b – выходная информация (результат работы системы);
    \item c – описание системы, ее критериев, свойств и т.д.;
    \item d – в случае некорректных данных или ошибок определения значений параметров системы – ошибки, возникшие в процессе работы;
    \item e – модель устаревания ИС;
    \item f – рассчитанная эффективность системы, а также предложенное решение о реинжиниринге ИС;
    \item g – графическое отображение результатов работы системы;
    \item h – запрос на создание отчетов.
\end{itemize}

\subsection{Принцип связности и модульности}
Любая часть системы должна рассматриваться со всеми своими связями с окружающими ее объектами, как внешними по отношению ко всей системе, так и внутренними – другими элементами системы.

Если некоторая подсистема имеет связи только с внешней средой, то есть смысл реализовать ее в виде отдельной системы.

Подсистема, не связанная ни с внешней средой, ни с другой подсистемой, является избыточной и должна быть удалена из системы.

В соответствии с принципом модульности, выделение модулей системы, если таковые имеются, продуктивно и оправдано.

Под модулями здесь понимаются относительно автономные и достаточно простые блоки, выполняющие ограниченный набор функций.

Модуль в отличие от подсистем, которые имеют нерегулярную структуру и, как правило, несут определенную функцию, частично отражающую функцию системы, имеет регулярную структуру и характерные для него внутренние и внешние связи.

Принцип указывает на возможность вместо части системы исследовать совокупность ее входных и выходных воздействий.

Выделению модулей соответствует декомпозиция сложной задачи на множество более простых подзадач.

Принцип модульности для разрабатываемой системы поясняется с помощью рисунка \ref{subsystem_scheme2}, описывающего разбиение на модули системы графического отображения и совместной работы.

\addimg{subsystem_scheme2}{0.6}{Иллюстрация принципа модульности на основе подсистемы графического отображения и совместной работы}{subsystem_scheme2}

\subsection{Принцип функциональности}
Принцип определяет первичность функции по отношению к структуре, так же, как цель первична для функции. 
Другими словами, цель определяет функции системы, а функции определяют ее структуру — совокупность элементов с их связями. 
Структура же, в свою очередь, определяет параметры системы. 

В случае придания системе новых функций полезно пересматривать ее структуру, а не пытаться втиснуть новую функцию в старую схему. 
Поскольку выполняемые функции составляют процессы, то целесообразно рассматривать отдельно процессы, функции, структуры.

Функции подсистем приведены в пункте 2.1.

Морфологическая таблица функций системы и функций назначения подсистем приведена в таблице \ref{table:morphological}.

\begin{table}[H]
    \centering
    \caption{Морфологическая таблица}\label{table:morphological}
    \begin{tabular}{|c|c|c|c|c|}
    \hline Функции & Подсистема 1 & Подсистема 2 & Подсистема 3 & Подсистема 4 \\
    \hline $\Phi_1$ & + & + &   &  \\
    \hline $\Phi_2$ &  & + & + &  \\
    \hline $\Phi_3$ &  & + & + & + \\
    \hline $\Phi_4$ &  &  & + & + \\
    \hline $\Phi_5$ &  & + &  &  + \\
    \hline
    \end{tabular}
\end{table}

В данной таблице знаком «+» обозначены функции, которые реализуются для каждой из подсистем.

Детализация функций подсистемы на примере подсистемы анализа и принятия решений:
\begin{itemize}
    \item нахождение коэффициентов морального устаревания системы;
    \item определение значения эффективности в заданный момент времени;
    \item принятие решения о реинжиниринге системы.
\end{itemize}

\subsection{Принцип cочетания централизации и децентрализации}
Степень централизации должна быть минимальной, обеспечивающей выполнение поставленной цели. 
Соотношение централизации и децентрализации определяется уровнями, на которых вырабатываются и принимаются управленческие решения.

Недостатком децентрализованного управления является увеличение времени адаптации системы, которое существенно влияет на функционирование системы в быстро меняющихся средах. 
То, что в централизованных системах можно сделать за короткий промежуток времени, в децентрализованной системе будет осуществляться весьма медленно. 
Недостатком централизованного управления является сложность управления из-за огромного потока информации, подлежащей переработке в старшей системе управления. 

В медленно меняющейся обстановке децентрализованная часть системы успешно справляется с адаптацией поведения системы к среде и с достижением глобальной цели системы за счет оперативного управления, а при резких изменениях среды осуществляется централизованное управление по переводу системы в новое состояние.

Например, можно выполнить декомпозицию подсистемы ввода таким образом:
\begin{itemize}
    \item подсистема ручного ввода параметров системы;
    \item подсистема webhooks для автоматического приема параметров системы (подсистема мониторинга);
    \item подсистема графического вывода.
\end{itemize}

Децентрализация данных подмодулей позволяет эффективнее осуществлять сбор и первичную обработку данных, необходимых для последующего анализа.

\subsection{Принцип развития}

Учет изменяемости системы, ее способности к развитию, адаптации, расширению, замене частей, накапливанию информации. 
В основу синтезируемой системы требуется закладывать возможность развития, наращивания усовершенствования. 
Обычно расширение функций предусматривается за счет обеспечения возможности включения новых модулей, совместимых с уже имеющимися. 
С другой стороны, при анализе принцип развития ориентирует на необходимость учета предыстории развития системы и тенденций, имеющихся в настоящее время, для вскрытия закономерностей ее функционирования.

Одним из способов учета этого принципа разработчиками является рассмотрение системы относительно ее жизненного цикла. 
Условными фазами жизненного цикла являются: проектирование, изготовление, ввод в эксплуатацию, эксплуатация, наращивание возможностей (модернизация), вывод из эксплуатации (замена), уничтожение.

Отдельные авторы этот принцип называют принципом изменения (историчности) или открытости. 
Для того чтобы система функционировала, она должна изменяться, взаимодействовать со средой.

Разрабатываемая система напрямую связана с реализацией данного принципа. 
В ходе анализа процесса устаревания производится принятие управленческого решения о модернизации(реинжиниринге) анализируемой системы, либо о выводе из эксплуатации. 
Также определяются такие параметры, как длительность эксплуатации и наиболее экономически целесообразный момент принятия такого решения. 

Проектируемая система может быть развита:
\begin{itemize}
    \item переходом на более современное оборудование;
    \item внедрением поддержки систем со сложной структурой;
    \item применением дополнительных математических методов для анализа и прогнозирования надежности;
    \item введением расширенного мониторинга исследуемой системы и т.д.
\end{itemize}

В данном разделе выполнен системотехнический анализ, в ходе которого разрабатываемый процесс рассмотрен с точки зрения принципов системного анализа. 
Выделены функции, подсистемы и структура процесса.

% На данный момент при анализе и синтезе сложных программных и аппаратных систем все чаще используется системный подход.
% Важным моментом для системного подхода является определение структуры системы --- совокупности связей между элементами системы, отражающих их взаимодействие.

% Принципы системного анализа --- это положения общего характера, являющиеся обобщением опыта работы человека со сложными системами.
% Пренебрежение принципами при проектировании любой нетривиальной технической системы, непременно приводит к потерям того или иного характера, от увеличения затрат в процессе проектирования до снижения качества и эффективности конечного продукта.

% Системный анализ выполнен в соответствии с \cite{sys-analyz}.

% \subsection{Принцип конечной цели} \label{goal}

% Принцип конечной цели --- основополагающий принцип системного анализа.
% Конечная цель имеет абсолютный приоритет, в системе все должно быть подчинено достижению конечной цели.
% Принцип имеет несколько правил:
% \begin{itemize}
%   \item для проведения системного анализа необходимо в первую очередь, сформулировать цель функционирования системы, так как расплывчатые, не полностью определенные цели влекут за собой неверные выводы;
%   \item анализ системы следует вести на базе первоочередного уяснения основной цели исследуемой системы, что позволит определить ее существенные свойства, показатели качества и критерии оценки;
%   \item при синтезе систем любая попытка изменения или совершенствования должна быть в первую очередь рассмотрена с позиции его полезности в достижении конечной цели;
%   \item цель функционирования искусственной системы задается, как правило, системой, в которой исследуемая система является составной частью.
% \end{itemize}

% В соответствии с данным принципом должна быть четко сформулирована конечная цель --- назначение проектируемой системы и сформирован список функций, которые должна выполнять система.

% Цель проектирования --- разработка системы безопасности облачной среды.
% Список функций проектируемой системы:
% \begin{itemize}
%   \item Ф1 --- авторизация и аутентификация пользователей;
%   \item Ф2 --- сетевая защита;
%   \item Ф3 --- идентификация и обработка инцидентов связанных с безопасностью;
%   \item Ф4 --- предоставление доступа к услугам;
%   \item Ф5 --- мониторинг.
% \end{itemize}

% \subsection{Принцип единства}

% Принцип единства --- это совместное рассмотрение системы как целого и как совокупности частей.
% Принцип ориентирован на декомпозицию с сохранением целостных представлений о системе.
% Система должна рассматриваться и как целое, и как совокупность элементов.
% Расчленение системы необходимо производить, сохраняя целостное представление о системе.
% Принцип подразумевает выделение подсистем, композиция которых в совокупности со связями позволяет выполнять все функции проектируемой системы, определяет ее структуру и может быть рассмотрена как единая, целостная система.

% На основании функций проектируемой системы, представленных выше, в ней можно выделить следующие подсистемы:
% \begin{enumerate}
%   \item подсистема аутентификации;
%   \item подсистема авторизации;
%   \item подсистема сетевой защиты;
%   \item подсистема проверки целостности данных.
% \end{enumerate}

% На рис. \ref{subsys} представлена схема взаимодействия между подсистемами.
% \addimg{subsys}{0.8}{Взаимодействие между подсистемами и их связь с окружающей средой}{subsys}

% Обозначения, приведенные на рис. \ref{subsys} требуют пояснения:
% \begin{itemize}[label={}]
%   \item a --- информация, предоставляемая пользователем передается на сервер;
%   \item b --- выходная информация (результат выполнения);
%   \item c --- проверка корректности переданных данных;
%   \item d --- в случае некорректных данных авторизации, возвращается управление к подсистеме аутентификации;
%   \item e --- предоставление доступов к инфраструктуре;
%   \item f --- возврат информации о правах доступа пользователя;
%   \item g --- обеспечение целостности данных инфраструктуры;
%   \item h --- обеспечение целостности данных пользователя.
% \end{itemize}

% \subsection{Принцип связности и модульности}

% Любая часть системы должна рассматриваться со всеми своими связями с окружающими ее объектами, как внешними по отношению ко всей системе, так и внутренними --- другими элементами системы.
% Если некоторая подсистема имеет связи только с внешней средой, то есть смысл реализовать ее в виде отдельной системы.
% Подсистема, не связанная ни с внешней средой, ни с другой подсистемой, является избыточной и должна быть удалена из системы.

% В соответствии с принципом модульности, выделение модулей системы, если таковые имеются, продуктивно и оправдано.
% Под модулями здесь понимаются относительно автономные и достаточно простые блоки, выполняющие ограниченный набор функций.
% Модуль в отличие от подсистем, которые имеют нерегулярную структуру и, как правило, несут определенную функцию, частично отражающую функцию системы, имеет регулярную структуру и характерные для него внутренние и внешние связи.
% Принцип указывает на возможность вместо части системы исследовать совокупность ее входных и выходных воздействий.
% Выделению модулей соответствует декомпозиция сложной задачи на множество более простых подзадач.

% Принцип модульности для разрабатываемой системы поясняется с помощью рис. \ref{modules}, описывающего разбиение на модули системы аутентификации.

% \addimg{modules}{0.65}{Принцип модульности на примере подсистемы аутентификации}{modules}

% Излишняя детализация не требуется, поэтому остальные системы на модули принято решение не разбивать.

% \subsection{Принцип функциональности}

% Принцип определяет первичность функции по отношению к структуре, так же, как цель первична для функции.
% Другими словами, цель определяет функции системы, а функции определяют ее структуру --- совокупность элементов с их связями.
% Структура же, в свою очередь, определяет параметры системы.
% В случае придания системе новых функций полезно пересматривать ее структуру, а не пытаться втиснуть новую функцию в старую схему.
% Поскольку выполняемые функции составляют процессы, то целесообразно рассматривать отдельно процессы, функции, структуры.

% Функции подсистем приведены в пункте \ref{goal}.

% Матрица инциденций функций системы и функций назначения подсистем приведена в табл. \ref{inc-matrix}.
% \begin{table}[H]
%   \caption{Матрица инциденций}\label{inc-matrix}
%   \begin{tabular}{|c|c|c|c|c|c|}
%   \hline \multirow{2}{*}{Функции} & \multicolumn{4}{|c|}{Подсистемы} & \multirow{2}{*}{Инфраструктура} \\
%   \cline{2-5} & 1 & 2 & 3 & 4 & \\
%   \hline Ф1 & + & + & & & + \\
%   \hline Ф2 & & & + & & + \\
%   \hline Ф3 & & + & & & + \\
%   \hline Ф4 & + & + & + & + & + \\
%   \hline Ф5 & + & + & + & + & + \\
%   \hline
%   \end{tabular}
% \end{table}

% В матрице инциденций знаком <<+>> обозначены функции, которые реализуются для каждой из подсистем.

% Детализация функций подсистемы на примере подсистемы аутентификации:
% \begin{enumerate}
%   \item обеспечение возможности ввода данных (веб-интерфейс или API);
%   \item обеспечение шифрованного канала для передачи данных;
%   \item сверка полученных данных с внутренней базой;
%   \item возможность восстановления доступов с помощью двухфакторной аутентификации.
% \end{enumerate}

% Входными данными для подсистемы является информация о пользователе, а выходными --- подтвержденный вход пользователя.

% \subsection{Принцип иерархии}

% Разработка иерархий классов является нетривиальной задачей.
% Грамотно спроектированные иерархии классов позволяют создавать высокоэффективные системы, плохо спроектированная иерархия приводит к созданию сложных и запутанных систем.

% Выполнение принципа иерархичности для разрабатываемой системы на примере подсистемы проверки целостности данных проиллюстрировано на рис. \ref{hierarchy}.
% \addimg{hierarchy}{1}{Принцип иерархии на примере подсистемы обеспечения целостности данных}{hierarchy}

% \subsection{Принцип сочетания централизации и децентрализации}

% Степень централизации должна быть минимальной, обеспечивающей выполнение поставленной цели.
% Соотношение централизации и децентрализации определяется уровнями, на которых вырабатываются и принимаются управленческие решения.

% Недостатком децентрализованного управления является увеличение времени адаптации системы, которое существенно влияет на функционирование системы в быстро меняющихся средах.
% То, что в централизованных системах можно сделать за короткий промежуток времени, в децентрализованной системе будет осуществляться весьма медленно.
% Недостатком централизованного управления является сложность управления из-за огромного потока информации, подлежащей переработке в старшей системе управления.
% В медленно меняющейся обстановке децентрализованная часть системы успешно справляется с адаптацией поведения системы к среде и с достижением глобальной цели системы за счет оперативного управления, а при резких изменениях среды осуществляется централизованное управление по переводу системы в новое состояние.

% Например, можно выполнить декомпозицию подсистемы сетевой защиты таким образом:
% \begin{enumerate}
%   \item подсистема работы виртуальной частной сети (\hyperlink{vpn}{VPN});
%   \item подсистема защиты от атак на отказ;
%   \item подсистема работы межсетевого экрана.
% \end{enumerate}

% Такое разбиение позволит реализовать полученные подмножества в виде отдельных модулей.

% \subsection{Принцип развития}

% Учет изменяемости системы, ее способности к развитию, адаптации, расширению, замене частей, накапливанию информации.
% В основу синтезируемой системы требуется закладывать возможность развития, наращивания усовершенствования.
% Обычно расширение функций предусматривается за счет обеспечения возможности включения новых модулей, совместимых с уже имеющимися.
% С другой стороны, при анализе принцип развития ориентирует на необходимость учета предыстории развития системы и тенденций, имеющихся в настоящее время, для вскрытия закономерностей ее функционирования.

% Одним из способов учета этого принципа разработчиками является рассмотрение системы относительно ее жизненного цикла.
% Условными фазами жизненного цикла являются: проектирование, изготовление, ввод в эксплуатацию, эксплуатация, наращивание возможностей (модернизация), вывод из эксплуатации (замена), уничтожение.

% Отдельные авторы этот принцип называют принципом изменения (историчности) или открытости.
% Для того чтобы система функционировала, она должна изменяться, взаимодействовать со средой.

% Проектируемая система может быть развита:
% \begin{itemize}
%   \item переходом на более гибкие файловые системы;
%   \item внедрением дополнительных технологий виртуализации;
%   \item увеличением штата системных инженеров;
%   \item введением расширенного мониторинга;
%   \item многоуровневой защитой от атак на отказ;
%   \item расширением памяти на системах хранения данных;
%   \item многоуровневой репликацией и резервным копированием.
% \end{itemize}

% \subsection{Принцип учета случайностей}

% Можно иметь дело с системой, в которой структура, функционирование или внешние воздействия не полностью определены.

% Сложные системы не всегда подчиняются вероятностным законам.
% В таких системах можно оценивать <<наихудшие>> ситуации и проводить рассмотрение для них.
% Этот способ обычно называют методом гарантируемого результата, он применим, когда неопределенность не описывается аппаратом теории вероятностей.

% События и действия, некорректные с точки зрения правил функционирования системы:
% \begin{itemize}
%   \item попытка входа без использования двухфакторной аутентификации;
%   \item сбой в работе аппаратуры, в частности устройств хранения данных;
%   \item сбой сети в в пределах дата-центра или на магистрали;
%   \item отсутствие своевременной работы системных инженеров;
%   \item ошибки и уязвимости в используемых программных платформах.
% \end{itemize}

% Кроме того, необходимо вести контроль успешности и целостности проведения операций с компонентами системы, данными пользователей и корректно обрабатывать исключения, возникновение которых возможно в процессе работы системы.

% Перечисленные выше принципы обладают высокой степенью общности.
% Такая интерпретация может привести к обоснованному выводу о незначимости какого-либо принципа.
% Однако, знание и учет принципов позволяет лучше увидеть существенные стороны решаемой проблемы, учесть весь комплекс взаимосвязей, обеспечить системную интеграцию.

\clearpage
