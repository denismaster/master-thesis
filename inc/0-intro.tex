\anonsection{Введение}

\textbf{Актуальность темы.}
Важность исследования процесса деградации сложных технических объектов (СТО) обусловлена следующими факторами: 
быстрым совершенствованием технологий, вплоть до появления принципиально новых; ростом сложности требований к техническим системам и ростом сложности самих систем. 
Инженерам необходимо иметь механизмы принятия решений по ее развитию с учетом технической, информационной и программной составляющих.
При этом данные механизмы должны быть доступны на протяжении всего жизненного цикла системы(ЖЦ), в особенности – на этапе эксплуатации и деградации.

Согласно ISO/IEC 2382-1:1993 \cite{ISO}, эффективность — это сложное интегральное свойство системы, характеризующее ее способность выполнять поставленную цель в заданных условиях использования и с определенным качеством. 
Данный интегральный показатель может быть рассчитан на основе структуры системы, а его значение варьируется на протяжении всего ЖЦ вследствие неоднородной структуры системы. 

Разнородность и слабая формализуемость элементов технических систем может привести к резким скачкам значений интегрального показателя эффективности (ИПЭ). 
Данные скачки могут означать как резкое повышение эффективности, так и непредсказуемый спад. 
Для эффективного реагирования механизмы принятия решений должны обладать возможностью прогнозирования данных изменений значения ИПЭ. 
С другой стороны, существует задача определения самой структуры системы в зависимости от заданных показателей.

Существующие системы и технологии, к сожалению, не предлагают готовых решений для вышеупомянутых проблем.
В большинстве своем авторы либо указывают простейшие способы анализа процессов ЖЦ системы, в особенности процесса деградации, либо не упоминают его вообще. 
Таким образом, данная тема является актуальной.

\textbf{Цели и задачи работы.} 
Целью данной выпускной квалификационной работы является создание системы поддержки принятия решений при анализе процесса деградации комплексных систем.
Для достижения указанной цели поставлены следующие задачи: 
\begin{itemize}
  \item сформулировать постановку задачи, выполнить ее формализацию;
  \item рассмотреть существующие подходы для решения поставленных проблем путем анализа литературных источников;
  \item выполнить системный анализ данной предметной области, выполнить декомпозицию будущей системы, определить ее структуру и выделить подсистемы в ней, определить связи между ними, а также между системой и внешней средой;
  \item разработать систему поддержки принятия решений(СППР) с использованием современных Web-технологий, провести тестирование данной системы;
  \item провести исследование деградационных процессов в различных сложных технических объектах.
\end{itemize}

\textbf{Предмет и объект исследования.}
В данной выпускной квалификационной работе объектом исследования является сложный технический объект (СТО).
Предметом исследования в данной работе является процесс деградации СТО.

\textbf{Структура работы.}
Данная работа состоит из пояснительной записки, которая включает в себя введение, четыре раздела, заключение, список использованной литературы, приложений, и программного модуля на электронном носителе.

В первом разделе представлен анализ данной проблемы и ее предметной области. 
Производится анализ требований к системе, определены функциональные и нефункциональные требования. 
Выбрана соответствующая модель качества системы. 

Во втором разделе показано проектирование системы, подсистем с использованием принципов системного подхода. 
Исследована архитектура системы. 
Описывается математическая модель интерполяции процесса деградации с использованием сплайнов, а также модели решения поставленных задач. 
Также произведен выбор средств и методологий для решения поставленных задач с использованием метода анализа иерархий.

В третьем разделе приводится процесс разработки системы и базы данных, а также ее тестирование; производится сравнение методов аппроксимации на основе сплайнов и других методов.

В четвертом разделе производится исследование процесса деградации сложных технических объектов с использованием разработанной СППР.

В заключении дается вывод о проделанной работе, резюмируются результаты и производится описание дальнейших вариантов развития системы.

В приложениях приводится исходный код разработанной информационной системы.


% Облачные услуги --- это способ предоставления, потребления и управления технологией.
% Данный тип услуг выводит гибкость и эффективность на новый уровень, путем эволюции способов управления, таких как непрерывность, безопасность, резервирование и самообслуживание, которые соединяют физическую и виртуальную среду.

% Для эффективной работы облачной инфраструктуры требуется эффективная структура и организация.
% Небольшая команда из специалистов и бизнес-пользователей может создать обоснованный план и организовать свою работу в подобной инфраструктуре.
% Данная выделенная группа может намного эффективнее построить и управлять нестандартной облачной инфраструктурой, чем если компании будут просто продолжать добавлять дополнительные сервера и сервисы для поддержки центра обработки данных (\hyperlink{dc}{ЦОД}).

% Развитие информационного мира движется в сторону повсеместного распространения облачных вычислений, их технологий и сервисов.
% Очевидные преимущества данного подхода \cite{telecom-world}:
% \begin{itemize}
%   \item снижение затрат --- отсутствие необходимости покупки собственного оборудования, программного обеспечения (\hyperlink{soft}{ПО}), работы системного инженера;
%   \item удаленный доступ --- возможность доступа к данным облака из любой точки мира, где есть доступ в глобальную сеть Интернет;
%   \item отказоустойчивость и масштабируемость --- изменение необходимых ресурсов в зависимости от потребностей проекта, техническое обслуживание оборудования лежит на плечах облачного провайдера.
% \end{itemize}

% В связи с этим можно сделать вывод, что основные недостатки облачных вычислений сводятся к информационной безопасности.
% Такого мнения придерживаются многие крупные информационные компании, что в некоторой степени препятствует более стремительному развитию рынка облачных сервисов.

\clearpage
